%% LyX 2.0.0 created this file.  For more info, see http://www.lyx.org/.
%% Do not edit unless you really know what you are doing.
\documentclass[12pt,a4paper,english]{article}
\usepackage{mathptmx}
\usepackage[scaled=0.9]{berasans}
\usepackage[scaled=0.8]{beramono}
\renewcommand{\familydefault}{\rmdefault}
\usepackage[T1]{fontenc}
\usepackage[utf8]{inputenc}
\usepackage{fancyhdr}
\pagestyle{fancy}
\usepackage{color}
\usepackage{babel}
\usepackage{wrapfig}
\usepackage{textcomp}
\usepackage{amsthm}
\usepackage{amsmath}
\usepackage{amssymb}
\usepackage{graphicx}
\usepackage[unicode=true,pdfusetitle,
 bookmarks=true,bookmarksnumbered=false,bookmarksopen=false,
 breaklinks=false,pdfborder={0 0 0},backref=false,colorlinks=true]
 {hyperref}

\makeatletter

%%%%%%%%%%%%%%%%%%%%%%%%%%%%%% LyX specific LaTeX commands.
\pdfpageheight\paperheight
\pdfpagewidth\paperwidth

\providecommand{\LyX}{\texorpdfstring%
  {L\kern-.1667em\lower.25em\hbox{Y}\kern-.125emX\@}
  {LyX}}

%%%%%%%%%%%%%%%%%%%%%%%%%%%%%% Textclass specific LaTeX commands.
\numberwithin{equation}{section}
\numberwithin{figure}{section}

%%%%%%%%%%%%%%%%%%%%%%%%%%%%%% User specified LaTeX commands.
% for: LaTeX DogWagger version=`4.0.4' fileTarget=`Dogwagger404.pl' startComment=`#'
%========================================================================
% LaTeX Dogwagger takes LaTeX files and pulls out verbatim comments,    %
% concatenating the text enclosed in these verbatim comments            %
% into executable code. Copyright (C) J van Schalkwyk, 2005.            %
% LaTeX Dogwagger is made available under the GNU Public Licence (GPL). %
% To NOT include a particular verbatim section, a comment line          %
%   containing the text: DogWagger dogsAllowed=`no' must precede *ANY*  %
%   line containing the text \begin{verbatim} !                         %
%===============================================================
\hyphenation{fileTarget endFile}
\usepackage{titlesec}
\hyphenation{fileTarget endFile}
\titleformat{\section}{\LARGE\sffamily}{\thesection}{1em}{}
\titleformat{\subsection}{\Large\sffamily}{\thesubsection}{0.5em}{}
\titleformat{\subsubsection}{\large\sffamily}{\thesubsubsection}{0.5em}{}

\makeatother

\begin{document}

\title{\textsf{\Huge \LaTeX{}\ Dogwagger }\\
\textsf{\Huge{} }\textsf{\footnotesize {[}A different approach to documentation{]}
}\textsf{\normalsize }\\
\textsf{\normalsize{} }\textsf{\scriptsize Version 4.0.4}}


\author{J.M. van Schalkwyk%
\thanks{With some help! See the change log.%
}}

\maketitle
\begin{wrapfigure}[5]{o}[2in]{3cm}%
\vskip -4.2in \hskip -4.6in\includegraphics[width=5.5cm,height=4cm]{images/Dogwagger}\end{wrapfigure}%


\textsf{\tableofcontents{}}\newpage{}


\section{Introduction}

In a single line, you can change for the better how you document your
computer code. Here's that line:\medskip{}


\textsf{\textcolor{blue}{\% DogWagger version=\textrm{\large`}4.0.4' fileTarget=\textrm{\large`}MYFILE.pl'}}\medskip{}




Inserted at the start of a \LaTeX{} file, this comment line allows
you to combine your code and documentation \emph{without} having to
learn the use of new tools. This example generates Perl but, with
a little help from Dogwagger, a single \LaTeX{} file can spawn multiple
different files in multiple languages --- html, PHP, Javascript, C,
C++, SQL, R, Erlang, and so forth.


\subsection*{What's the catch?}

The only catch is that you must be familiar with \LaTeX{}, or (for
those who are allergic to natural rubber), you must have tried the
almost-human-friendly alternative, \LyX{}.%
\footnote{\LyX{} users must however be familiar with the term {}``evil red
text''. %
} If you're terrified by the idea of running a program from the command
line, then Dogwagger may also not be for you. 

Dogwagger is freely available under the GNU Public Licence (See Appendix~\ref{sec:Appendix:-GNU-GPL}),
and will work with a simple, free text editor (like Notepad, vim,
Emacs) or specific \LaTeX{}-friendly editors like \LyX{} and Kile. 


\subsection*{How does it work?}

You simply embed all of your code within the \LaTeX{} document as
\emph{verbatim} statements, and submit the .tex file to DogWagger,
which pulls out the code and turns it into a program. The living proof
of this is the document you're reading, which not only describes but
\emph{contains} the entire source code of Dogwagger!


\subsection*{Why Dogwagger?}

It's simple. Other documentation managers have problems:
\begin{itemize}
\item Complex dependencies;
\item A steep learning curve; or
\item The need to adopt a new programming approach (e.g. noweb).%
\footnote{A new approach may not be an entirely bad thing, but the resulting
fragmentation may not be kind to those used to reading {}``more normal''
code. %
}
\end{itemize}

\subsection*{Why the name?}

Conventional documentation of computer code often looks tacked on,
resembling the stumpy, customarily docked tail of a large rottweiler.
Dogwagger tries to address this problem by integrating the documentation
and the program. The program becomes something which is pulled out
of the documentation, rather than the other way around.

For example, in the great programming language Perl, there's a convention
that allows you to mark sections of the program with an equals sign,
followed by a name. All of the subsequent code is ignored by Perl
until a magic line beginning with the expression \verb+=cut+ is encountered.
A separate program can then be used to pull out the \emph{cut} sections
and assemble them into some sort of documentation. This approach is
called POD, or `Plain Old Documentation'. Dogwagger, although it is
written in Perl, is more sophisticated, as it works the other way
around. The tail wags the dog.


\subsection*{Why \protect\LaTeX{} \& Perl?}

\LaTeX{} is a good way to produce documents that are both functional
and fairly elegant, especially if you're working on a large project
and need good technical documentation. It's the document preparation
system of choice for mathematical and astrophysical journals, and
has been tried and tested over decades. Perl is powerful and available
on almost any platform you care to mention.


\subsection*{How do I run Dogwagger?}

You will need Perl installed on your system (most UNIX and Linux systems
will have it; with MS Windows, consider Strawberry Perl or ActivePerl). 

Open up a console box (otherwise known as a `terminal' or the `command
line', e.g. Konsole in Ubuntu, or a DOS box in MS Windows), change
to the directory where you've put the file Dogwagger404.pl, and type
in:

\medskip{}


\hskip 1cm\textsf{\textbf{perl Dogwagger404.pl MYFILE.tex}}\medskip{}


\ldots{} for Dogwagger to work its magic. Type the following from
the command line: \medskip{}


\hskip 1cm \textsf{\textbf{perl Dogwagger404.pl\quad{}-\,-help}}

\medskip{}


\ldots{} for a full list of options, or simply read on. \newpage{} 


\subsection{Dogwagger: Advantages \& disadvantages}

The few disadvantages have already been mentioned. In more detail
the advantages are:
\begin{itemize}
\item Program code and documentation are seamlessly integrated; 
\item Updates are concurrent. You can update program and documentation at
one go, and generate program \emph{or} documentation by simply submitting
the same file to either PDF\LaTeX{}\ or Dogwagger; 
\item Binary code can be integrated (as required) with other code, without
needing fancy tools or many different types of file; 
\item As Dogwagger is available under the GNU public licence, it'll aways
be freely available. You can download the source code of this file
(Dogwagger404.tex) from the dogwagger page at the website \href{http://www.anaesthetist.com/mnm/dogwagger/index.htm}{anaesthetist.com}.
\end{itemize}

\subsection{GNU Public Licence}

\textbf{NB.} This program is distributed under the Gnu Public Licence
(GPL). A copy should accompany any distribution. For details of the
GPL, see Appendix~\ref{sec:Appendix:-GNU-GPL}, at the end of this
document. This program is free software; you can redistribute it and/or
modify it under the terms of the GNU General Public License as published
by the Free Software Foundation; either version 2 of the License,
or (at your option) any later version. This program is distributed
in the hope that it will be useful, but WITHOUT ANY WARRANTY; without
even the implied warranty of MERCHANTABILITY or FITNESS FOR A PARTICULAR
PURPOSE. See the GNU General Public License for more details.

You should have received a copy of the GNU General Public License
along with this program; if not, write to the Free Software Foundation,
Inc., 59 Temple Place - Suite 330, Boston, MA 02111-1307, USA.

\newpage{} \hypertarget{below}{} 


\section{How to use Dogwagger}

If you read the \LaTeX{}\ source for \emph{this file}, (available
as both .lyx and .tex documents) then you'll see Dogwagger in action!
But to get a bit more than a flavour read the following documentation
thoroughly. The basic idea is:
\begin{enumerate}
\item Create your \LaTeX{}\ documentation; 
\item At will, insert sections of program within the documentation as \verb+\verbatim+
comments --- these chunks will eventually all be concatenated into
one program file! 
\item Dress things up a little. 
\end{enumerate}
That's really it. The dressing up is very easy indeed. You've already
encountered the most important piece of `dressing up', the Dogwagger
title line:

\medskip{}
\textsf{\textcolor{blue}{\% Dogwagger version=\textrm{\large`}4.0.4' fileTarget=\textrm{\large`}foo.pl'
}}\medskip{}


If such a line isn't present, and you submit a file for Dogwagger
to parse, then she will complain bitterly. You'll also get warnings
if the version number is wrong. Note the backticks (\textrm{\large`})
at the start of \textrm{\large`}quoted items\verb+'+. The \verb+fileTarget+
instruction is self-explanatory.

There are a two other parameters you will almost certainly use on
this important title line. Examples are:
\begin{itemize}
\item \textsf{\textcolor{blue}{startComment=\textrm{\large`}\#' }}
\item \textsf{\textcolor{blue}{noWarn=\textrm{\large`}yes' }}
\end{itemize}
The \verb+startComment+ option specifies how lines will be commented
out. This is important because DogWagger normally writes a few lines
at the start of a generated file saying where the file came from and
when it was created. In addition, separate sections in the generated
program code are separated by comments. The default comment character
is \verb+#+.

Different programming languages use different character sequences
to signal a comment line --- for example, C++ uses a double slash
\verb+//+, and Perl uses \verb+#+! But we can even implement the
old-fashioned C \verb+/*+ comment style \verb+*/+ with the following
combination of instructions:

\medskip{}
\textsf{\textcolor{blue}{\% Dogwagger startComment=\textrm{\large`}/*' endComment=\textrm{\large`}*/' [etc]
}}\medskip{}


By default, Dogwagger kindly warns you before it overwrites files,
but you can override this behaviour using \textsf{\textcolor{blue}{noWarn=\textrm{\large`}yes'}}\textcolor{blue}{{}
}.


\subsection{Suppress unwanted \emph{verbatim} sections}

If you don't want the next \verb+verbatim+ section to appear in code,
say:

\medskip{}
\textsf{\textcolor{blue}{\% Dogwagger dogsAllowed=\textrm{\large`}no'
}}\medskip{}



\subsection{Make multiple files}

Part of the way through your documentation, you may wish to terminate
the current code (file) you're creating, and start a new file. For
example, you may have discussed (and created) the main CPP program,
and now wish to do the same for the .H header file. This is pretty
easy. In the line \emph{immediately preceding} the next \verb+verbatim+
section, insert a commented line similar to the following:

\medskip{}
\textsf{\textcolor{blue}{\% Dogwagger newTarget=\textrm{\large`}foobar.h' startComment=\textrm{\large`}//'
}}\medskip{}


Remember to make the very first character of the line a `\%', so as
to comment the line out in \LaTeX{}. See how we use \textsf{\textcolor{blue}{newTarget}}
to remind ourselves that this isn't the first file specified in the
title line, but a subsequent one.


\subsubsection{Files within directories}

We deliberately don't encourage files generated by DogWagger to be
written to obscure locations. Files can however be written to an \emph{existing}
subdirectory thus:

\medskip{}
\textsf{\textcolor{blue}{\% Dogwagger newTarget=\textrm{\large`}foo/bar.c' startComment=\textrm{\large`}//'
}}\medskip{}


This statement creates the file \emph{bar.c} in the subdirectory \emph{foo}.
Always use a forward slash (a la Unix) not the DOS/MS Windows backslash.
Dogwagger will \emph{not} create the subdirectory. 


\subsection{Deferred code}

\label{deferred} You can \emph{defer} writing of a code section until
other sections on which it \emph{depends} have been written to the
output file. For example, when discussing SQL code, we might wish
to talk about the main table first, but in the final code we will
first want to define the minor tables on which the main table depends!

To make use of this facility, use the following commands:

\medskip{}
\textsf{\textcolor{blue}{\% Dogwagger dependsOn=\textrm{\large`}alpha'   
}}\medskip{}


\ldots{} or even, if something depends on several other sections:

\medskip{}
\textsf{\textcolor{blue}{\% Dogwagger dependsOn=\textrm{\large`}alpha,beta,gamma'      
}}\medskip{}


The name of the section depended on is then given by:

\medskip{}
\textsf{\textcolor{blue}{\% Dogwagger myName=\textrm{\large`}alpha'     
}}\medskip{}


When a dependsOn statement is encountered the assumption is made that
\emph{all} of the names depended on have \emph{not yet been defined}!
If any of them existed, then the smart user would simply leave them
out! 

An item can have entries for both \textsf{\textcolor{blue}{myName}}
and\textsf{\textcolor{blue}{{} dependsOn}}. In this case, the name
is kept pending until all \textsf{\textcolor{blue}{dependsOn}} blocks
have been created, at which point the item is written to output; then
only is the name of the item itself resolved.%
\footnote{Self-dependence will be unresolved, causing an error.%
}

It's important to realise that if two sections of code A and B are
deferred pending the writing of code C, then the only way to ensure
that B is written after A is to explicitly state that B depends on
A.


\subsection{Include binary files}

Because binary files, especially executable ones, are inscrutable
(and potentially harmful) you should generally avoid spreading them
around, but occasionally it may be necessary to include such a file
with your source code. Dogwagger meets this need by allowing UUencoded
files to be included in \verb+verbatim+ sections thus:

\medskip{}
\textsf{\textcolor{blue}{\% Dogwagger newTarget=\textrm{\large`}uudecode'   
}}\medskip{}


In other words, the only `filename' which is reserved is the case-sensitive\textsf{\textcolor{blue}{{}
uudecode}}. If this name is specified, then (as is usual for uunecoded
files) the filename is picked out of the subsequent uuencoded information,
located within the verbatim statement. There's a (uuencoded) uunecoding
program in Appendix~\ref{sec:Appendix:-UUencoding-program}.%
\footnote{In Ubuntu Linux, try {}``sudo apt-get install sharutils''.%
}

Because uuencoded text for a single file must be contained in one
\verb+verbatim+ section, and this file is written immediately, you
can write such a file from within the middle of creating other code.
(This means that, in contrast to \verb+newTarget+, you should not
specify uuencode as the \verb+fileTarget+ parameter). 


\subsection{Minor frills and restrictions}

In between the beginning and end of each `verbatim' section there
\emph{must} be at least one line of data. If you put a carriage return
immediately after the \verb+\begin{+\verb+verbatim}+ it will be ignored.
However, a carriage return just before an  \verb+\end{+\verb+verbatim}+
statement will be included in the output code. Play with this feature.

There are several little conveniences in DogWagger. You can label
the comment at the start of each section using a line like the following
\emph{immediately preceding} a \verb+verbatim+ section:

\medskip{}
\textsf{\textcolor{blue}{\% Dogwagger sectionTitle=\textrm{\large`}This is a BIG BOLD section'  
}}\medskip{}


We can do slightly more fancy things:

\medskip{}
\textsf{\textcolor{blue}{\% Dogwagger sectionTitle=\textrm{\large`}Foo: Section \$[SECTION]'
}}\medskip{}


\ldots{} which actually uses Dogwagger's internal section counter
to replace\linebreak{}
 \textsf{\textcolor{blue}{\$[SECTION] }}with the relevant section
number.


\subsubsection{Write code as a single line}

Occasionally it's convenient to write several lines of verbatim text
as a single line of output. Dogwagger to the rescue with the \textsf{\textcolor{blue}{oneLine=\textrm{\large`}yes'}}
command! 

Note that in this mode, trailing spaces count, but the leading spaces
on the next line are removed. All other whitespace is preserved \emph{as
is}.


\subsection{Make a debug version!}

Here's a command which you can \emph{only} include in the title line:

\medskip{}


\textsf{\textcolor{blue}{include=\textrm{\large`}everything' }}

\medskip{}


This wrinkle allows you to create two versions of code, a \emph{debug}
version, and a production version. By default, if you \emph{omit}
the above command from the title line, then the production version
is created. If you include it, then we make a debug version. And what's
the difference? Well, if Dogwagger encounters a line within a \verb+verbatim+
section which begins with the sequence:

\medskip{}
\textsf{\textcolor{blue}{+OPTIONAL}} \medskip{}


\ldots{} then by default all of the code  is \emph{omitted} until:

\medskip{}
\textsf{\textcolor{blue}{-OPTIONAL}} \medskip{}


is encountered later on. `Including everything' forces Dogwagger to
include this optional code.

In many languages, there are ways of creating debug versions, for
example the C++ \verb+#define+ followed by \verb+#ifdef+ and so
on, but our way is more explicit and simpler.%
\footnote{We considered having an optional parameter after the OPTIONAL statement
to allow multiple versions, but rejected this as extremely silly.%
}


\subsection{Use Dogwagger with \protect\LyX{}\label{sub:Use-Dogwagger-with-LyX}}

I believe that \LyX{}, the first {}``what you see is what you mean''
word-processor, is now mature enough to supplant other \LaTeX{} editors.
Put \verb+\begin+\verb+{verbatim}+ and \verb+\end+\verb+{verbatim}+
statements and contained code within {}``Evil Red Text'' boxes (ERT,
press Control+L). There are only two catches:
\begin{enumerate}
\item ERT is raw \LaTeX{}, so be careful to balance begin/end statements,
braces, \$ and so forth. (Also use \verb+\verb+ with great caution).
\item It's best to start the ERT with a comment line, before the actual\linebreak{}
 \textsf{\textcolor{blue}{\%~DogWagger}} line. This is because \LyX{}
may otherwise sometimes confuse Dogwagger by putting other characters
before the \% at the start of the\textsf{\textcolor{blue}{{} \%~DogWagger }}
line. 
\end{enumerate}
The very first Dogwagger line is best put in the \LaTeX{} preamble
(Document | Settings).

\newpage{}


\section{The source code of Dogwagger}

Here's the Dogwagger version 4.0.4 code. 

%
% DogWagger sectionTitle=` Header (Section $[SECTION])'
{\footnotesize\begin{verbatim}
#!/usr/local/bin/perl -w
use strict;

  ###################################################################
  # This program is distributed under the Gnu Public Licence (GPL).
  # You should have received a copy of the GNU General Public License
  # along with this program; if not, write to the Free Software
  # Foundation, Inc., 59 Temple Place - Suite 330, 
  # Boston, MA  02111-1307, USA.
  ###################################################################

my $ERRCOUNT=0;        # global. Ugh.
my $LINECOUNT=0;       # likewise
my $OPTN=0;            # global that supports +OPTIONAL 
my $MAJORVERSION = 4;  # * REMEMBER TO CHANGE FROM VERSION 4.0.4
my $MINORVERSION = 0;  # * REMEMBER TO CHANGE FROM VERSION 4.0.4
my $TV  = 4;           # * version 4.0.4 (TV = tinyversion)
\end{verbatim}}


\subsection{Initialisation}

We get the current (system) time (Section~\ref{sub:Read-the-time}),
and then create several important arrays used for processing of deferred
code. 

%
% DogWagger sectionTitle=` Initialisation (Section $[SECTION])'
{\footnotesize \begin{verbatim}
my $TODAY = &GetLocalTime();
my @CHILDREN;
my @DEPENDENCIES;
my @PENDINGNAME;
my $DEFAULTNEWLINE = "\n";
my $NEWLINE = $DEFAULTNEWLINE;
\end{verbatim}}

The deferred arrays are CHILDREN (an array of sections which depend
on other named sections, and haven't yet been written to file), DEPENDENCIES
which stores the corresponding names \emph{depended on}, and PENDINGNAME,
which stores the name of each child, if that child has a name.

Each child contains multiple lines which will only be written to file
when all of the dependencies of the child have been satisfied. As
each name is encountered, it is removed from each dependency list
where it occurs. If a dependency list becomes \emph{empty} in this
process, the corresponding item is written to output.

The variable \verb+$NEWLINE+ is a global that lets us specify the
newline character(s) throughout all files that are printed.


\subsection{Setup for file reentrancy \label{sub:Setup-for-file-reentrancy}}

As of Version v4.0.4 (tme), this code initialises the associative
array \texttt{\%APPENDTOFILE} that lets the program know whether to
append to a file or just overwrite it. This tiny amendment allows
you to re-specify a file multiple times --- each new invocation of
fileTarget=\ldots{} appends the subsequent code to the file!

%
% DogWagger sectionTitle=` Initialisation (Section $[SECTION])'
{\footnotesize \begin{verbatim}
my %APPENDTOFILE = (); # v4.0.4 (tme)
\end{verbatim}}

\newpage{}


\section{Startup}

 \vskip -1mmWe handle the command line \verb+--help+ option with
a help message:

%
% DogWagger sectionTitle=` Startup (Section $[SECTION])'
{\footnotesize\begin{verbatim}
my ($HelpMessage) = "\n      perl Dogwagger"
 . " $MAJORVERSION$MINORVERSION$TV.pl SOURCENAME.tex [log=WAGLOG.LOG]\n"
 . "         See: http://www.anaesthetist.com/mnm/dogwagger/\n"
 . "==================================================================\n"
 . "  *Picks out program code from within LaTeX verbatim statements*\n"
 . "        Within the .tex file, a DOGWAGGER LINE starts with \n"
 . "% Dogwagger \n"
 . "           ... followed on that line by one or more of:\n"
 . "   Option (example)                      Meaning\n"
 . " --------------------     ---------------------------------------\n"
 . " version=`4.0.4'      ... state optimal Dogwagger version to use\n"
 . " fileTarget=`FOO.BAR' ... the name of the very FIRST target file\n"
 . " [newLine=`\\n\\r']     ... replace newline characters with these.\n"
 . "       <<ONLY use the above 3 in the first DOGWAGGER LINE>>\n"
 . " newTarget=`NEW.BAR'  ... end off preceding file, start new file\n"
 . "     <<on starting a new file, all of the following options \n"
 . "       are reset, apart from startComment and endComment>>\n"
 . " startFile=`<?php\\n'  ... very first text in file, default is ''\n"
 . " endFile=`\\n?>'       ... final text in file, default is also ''\n"
 . " noHeader=`yes'       ... Dogwagger will now  not write a header\n"
 . " noTail=`yes'         ... omit all Dogwagger code at end of file\n"
 . " startComment=`<!--'  ... define how a Dogwagger comment starts;\n"
 . " endComment=`-->'     ... how comment ends, (default is nothing)\n"
 . " noWarn=`yes'         ... do NOT warn before overwriting a file.\n"
 . "    <<the above 7 can only be used on starting a new file>>\n"
 . " dogsAllowed=`no'     ... do NOT include following verbatim text\n"
 . " sectionTitle=`BLAH'  ... force use of this as new section title\n"
 . " noSections=`yes'     ... turn section titles on  (or off: `no')\n"
 . " oneLine=`yes'        ... in this block alone, concatenate lines\n"
 . "                          (with suppression of leading spaces)\n"
 . " dependsOn=`ALPHA'    ... defer writing text  until name defined\n"
 . " dependsOn=`A,B,C'    ... defer  until multiple blocks are named\n"
 . " myName=`ALPHA'       ... name a block  (this is case sensitive)\n"
 . " newTarget=`uudecode' ... just extract a single uuencoded file.\n"
 . "    <<See the documentation for +OPTIONAL code inclusions>>\n"
 . " *--------------------------------------------------------------*\n"
 . " Your LaTeX code must include, in its first 40 lines, a DOGWAGGER\n"
 . " LINE that describes  both the version and the first target file.\n"
 . " All subsequent LaTeX verbatim sections are appended, by default.\n"
 . " To modify behaviour, put a DOGWAGGER LINE immediately preceding\n"
 . " the first line of a verbatim statement, using the above options.\n"
 . " *--------------------------------------------------------------*\n"
 . "   Dogwagger 4.0.4 is Copyright (C) J van Schalkwyk, 2005--2012\n"
 . "      Made available under the GNU General Public Licence v2.\n"         
 . "=================================================================\n";
\end{verbatim}}

After printing a Dogwagger banner (and the date), we obtain the name
of the .tex source file from the command line (ARGV). If this is instead
a request for help, we print the help message and exit; otherwise
we validate the file name.

%
% DogWagger sectionTitle=` write header (Section $[SECTION])'
{\footnotesize \begin{verbatim}
my ($fido) = "";  # file name
print "\n\n                     =======================\n"
        .    "                     Dogwagger Version "
        .                  "$MAJORVERSION.$MINORVERSION.$TV \n"
        . "                     =======================\n";
print ("                       $TODAY\n");
$fido = $ARGV[0]; # command line
if ( $fido =~ /--help/i )
   { print $HelpMessage;
     exit;
   } else
   { print "                      (Try --help for help)\n"; };
if ( !(defined $fido) || ((length $fido) < 5))
   {  die "Error. Bad file name. "
          . "Please submit a valid name e.g. foo.tex";  };
\end{verbatim}}

We check for a log file (a second, optional, command-line \linebreak{}
argument \textbf{log=FOO.log}), and otherwise default to the name\linebreak{}
 WAGLOG.LOG. We open this log, and then invoke WagTheDog:

%
% DogWagger sectionTitle=` Open log file (Section $[SECTION])'
{\footnotesize\begin{verbatim}
my $filelog = $ARGV[1];   # format is: log=filename.xxx
if ($filelog =~ /log=(\w+\.\w+)/i )
   { $filelog = $1;       # pull out name
   } else
   { $filelog="WAGLOG.LOG";  
   };
open FILELOG, ">$filelog" or 
     die "*CRASH* Could not open LOG <$filelog> :$!\n"; 
print FILELOG "LaTeX DogWagger, Version "
      . "$MAJORVERSION.$MINORVERSION.$TV [$TODAY]";
&Caution ("INPUT file: <$fido>");
my $cSECTIONS= 0;         # 4 globals: count verbatim sections,
my $cFILES   = 1;         #   generated files (must be >=1),
my $cSKIP = 0;            #   sections skipped,
my $cOUST = 0;            #   and files overwritten. 
if ( (my $fail = WagTheDog($fido))
   ||($ERRCOUNT > 0)
   ) { print "\n*Problems: $fail, Errors=$ERRCOUNT\n"; };
$TODAY = &GetLocalTime();
&Caution( "Done [$TODAY]. Sections:$cSECTIONS($cSKIP skipped), "
      . "files:$cFILES ($cOUST overwritten).\n\n" );
exit;
\end{verbatim}}

The main function WagTheDog (§\ref{sec:The-principal-function}) returns
an error message if it fails. A global error count over zero also
indicates failure.

\newpage{}


\section{The principal function\label{sec:The-principal-function}}

As the name suggests, WagTheDog does the work. WagTheDog itself has
several ugly features%
\footnote{Some of these are a legacy from when DogWagger had a GUI, others are
just bad programming.%
} --- let's see how DogWagger copes!

%
% DogWagger sectionTitle=` WagTheDog (Section $[SECTION])'
 \begin{verbatim}
sub WagTheDog
{ my($fido);
  ($fido)=@_;
  my(@CHILDREN, @DEPENDENCIES, @PENDINGNAME);
  my $RETAINTARGETLINE = ''; # used for new file parameters
\end{verbatim}


\subsection{Preliminaries}

After some debugging statements and a check for the presence of a
filename string, we clear the various arrays, and set up \verb+myNam+,
which stores the name of the current block of code that is being written.
Once we've finished writing this block, we will resolve all of the
dependencies on \verb+myNam+.

%
% DogWagger sectionTitle=` WagTheDog:preliminaries (Section $[SECTION])'
 \begin{verbatim}
  @CHILDREN = ();
  @DEPENDENCIES = ();
  @PENDINGNAME = ();
    $CHILDREN[0]='';
    $DEPENDENCIES[0]='';
    $PENDINGNAME[0]='';
  my($myNam);   
\end{verbatim}

Let's open the source file, failing if this opening fails:

%
% DogWagger sectionTitle=` WagTheDog:open source (Section $[SECTION])'
 \begin{verbatim}
  $LINECOUNT = 0;
  my ($E1) = 0; 
    open FIDO, $fido 
         or $E1 = &GlobalError("Could not open source <$fido> :$!");
if ($E1)
   { return ($E1);  
   };
\end{verbatim}


\subsection{Find the header line}

Next, scan through the first n=100 lines for the header line:

%
% DogWagger sectionTitle=` WagTheDog:header line (Section $[SECTION])'
 \begin{verbatim}
  my($i) = 100;
while ($i > 0)
   { $_ = <FIDO>;
     $LINECOUNT ++;
   if ( /\%.* Dogwagger/i )
      { $i = 0; }; # force end
        $i --; 
   };
if (! $i)  # if DogWagger found, $i should be -1.
   { &Caution("DogWagger data not found in <$fido>");
      close FIDO;  
      return ('No data');
   };
  my($hotline) = $_;  # redundant  
  $NEWLINE = &GetNewLine($hotline); # determine newline code    
\end{verbatim}

We fail if the first Dogwagger line is not encountered. The only requirements
 are the presence of the term ` Dogwagger', and that the line is
commented out à la \LaTeX{}.


\subsection{Check the version}

We will soon check for version compatibility, and extract Dogwagger
command data into the variables defined below:

%
% DogWagger sectionTitle=` WagTheDog: check version (Section $[SECTION])'
 \begin{verbatim}
  my ($version, $DOGFILE, $startComment, $nowarn, $sft,
      $eft, $endComment, $nohead, $nosections, $notail); 
  # amended 2011-12-18
  my($majorVersion, $minorVersion);
\end{verbatim}

All variables apart from those pertaining to comments are reset to
default (null string) after the closure of the current file, so for
each file with starting and/or ending text, the values must be specified
anew! ReadHeader (Section~\ref{sec:Reading-the-header}) obtains
the values:

%
% DogWagger sectionTitle=` WagTheDog: ReadHeader (Section $[SECTION])'
 \begin{verbatim}
  my($MANDATORY); 

  ($version, $DOGFILE, $startComment, $nowarn, $MANDATORY,
   $sft, $eft, $endComment, $nohead, 
   $nosections, $notail) = &ReadHeader($hotline); 
   # nohead.. added 2011-12-18

if ($version == 0) # error
   { return ('Bad header');
   };

  $_ = $version;
  /(.+)\.(.+)\.(.+)/;  # pull out major and minor version numbers:
  $majorVersion = $1;
  $minorVersion = $2;  # ignore trivial version number = $3  
\end{verbatim}

We check the version specified against the current version of Dogwagger:

%
% DogWagger sectionTitle=` WagTheDog:confirm version (Section $[SECTION])'
 \begin{verbatim}
if ($majorVersion > $MAJORVERSION)
   { &Caution( 
     "Warning: DogWag(V$MAJORVERSION.$MINORVERSION) "
    ."won't support all features of V$majorVersion.$minorVersion");
   } else
   { 
   if (  ($majorVersion == $MAJORVERSION)
       &&($minorVersion > $MINORVERSION)
      )
      { &Caution( "Caution: minor version switch. "
                 ."Problems may abound!");
      };
   };
\end{verbatim}

Dogwagger emits appropriate warnings if either the major or minor
version numbers of Dogwagger are incompatible with those in the file
being translated. Trivial version numbers (the third part of the dotted
version number) are ignored.


\subsection{Open the target file}

We open the target file (Section~\ref{sub:Open-the-target}), using
the name provided, and fail if this fails.

%
% DogWagger sectionTitle=` WagTheDog: open target (Section $[SECTION])'
 \begin{verbatim}
  my ($ok, $wagline);
if (! OpenTargetFile($DOGFILE, $startComment, $fido, 
                     $nowarn, $sft, $endComment, $nohead))  
                     # nohead.. added 2011-12-18
   { return ("Could not open target <$DOGFILE>"); 
   };  #fail
\end{verbatim}


\subsection{Read source}

Now we're ready to read in the source file, and process it. Several
startup flags control interpretation, the most important being \verb+ishot+,
which determines whether we are actively writing lines, or just throwing
away \LaTeX{}\ text. 

%
% DogWagger sectionTitle=` WagTheDog: read source (Section $[SECTION])'
 \begin{verbatim}
  my($ishot, $hotdata, $chomper, $chomped, $hotline); 
  my($nodogs);
  my($SECTION) = 1;
  my ($SECTIONTITLE) = ''; # default is empty
 
  $ishot   = 0;
  $chomper = 0;  # default is OFF
  $chomped = 0;
  $nodogs  = 0; # default
\end{verbatim}


\subsection{An enormous \emph{while} statement}

A biig \verb+while+ statement surrounds everything, within which
we read each line in turn and process it. 

%
% DogWagger sectionTitle=` WagTheDog: main WHILE (Section $[SECTION])'
 \begin{verbatim}
while (1)            #         an enormous while statement ****
   { $_ = <FIDO>;
   if (! defined)    # exit. 
      { close FIDO;
        print FILELOG "\n  Line $LINECOUNT: "; # preliminary to closing
        &CloseDogFile($startComment, $eft, $endComment, $notail);     
        return(0); 
      }; 

     $_ = &ChompLine($_);
     $LINECOUNT ++;
   if (! $ishot) # if not writing
\end{verbatim}

In the above, ChompLine (Section~\ref{sub:ChompLine}) chomps the
terminal `newline' character(s) off the input line that is read from
the file handle \verb+FIDO+. 


\subsection{Not hot\label{sub:Not-hot}}

The following code deals with the case where we are not `hot', i.e.
not writing to a target file.

%
% DogWagger sectionTitle=` WagTheDog: not hot (Section $[SECTION])'
 \begin{verbatim}
      { 
      if ( /(.*)\\begin\{verbatim\}(.*)/ ) # if "begin verbatim":
         { $cSECTIONS ++;    # bump verbatim section count
         if (! $nodogs)
            { $hotdata = $2;   # amended 2011-12-19
            if ($1 !~ /\%/ ) # if verbatim not commented out
               { $ishot = 1; # turn on
                 $SECTION = &PrintSectionHeader($startComment, 
                    $SECTION, $endComment, $nosections,
                    $SECTIONTITLE);
                 print DOGFILE $hotdata;   # clumsy but explicit
               };
            };
\end{verbatim}

The above deals with the case where we've just encountered \verb+\begin{+\verb+verbatim}+.
We ensure that this statement hasn't been commented out in \LaTeX{},
and if not, print a section header (See §\ref{sub:Print-a-section-header}).
Otherwise, we check for a Dogwagger instruction:

%
% DogWagger sectionTitle=` WagTheDog: instruction? (Section $[SECTION])'
 \begin{verbatim}
         } else                               # NOT "begin verbatim":
         { my($depOn);                 
           $myNam = '';
           $depOn = '';
           $nodogs = 0;

         if (/^\s*\%\s*DogWagger/i)           # if IS DogWagger ***
            { &PrintLogLine ( "{wag} ");
            if ( /dogsAllowed=\`no\'/)        # dogs NOT allowed
               { $nodogs = 1;
                 $cSKIP ++;
                 print FILELOG "__skip__";
               } else                         # dogs are allowed
\end{verbatim}

If the instruction is {}``no dogs allowed'' we set this flag; otherwise
we look for the other common Dogwagger commands as follows.

%
% DogWagger sectionTitle=` WagTheDog:other commands (Section $[SECTION])'
 \begin{verbatim}
               { $wagline = $_;             # dogs are allowed **
               if ( ! /^(.*)newTarget=\`(.+?)\'(.*)$/ ) # revised 4.0.3
                  { 
                  if (/^(.*)dependsOn=\`(.+?)\'(.*)$/)
                     { $depOn = $2;
                       $_ = "$1$3";
                       print FILELOG "dependencies <$depOn>; ";
                     };

                  if (/^(.*)myName=\`(.+?)\'(.*)$/)
                     { $myNam = $2;
                       $_ = "$1$3";
                       print FILELOG "name=$myNam; ";
                     };

                  if (/^(.*)oneLine=\`yes\'(.*)$/) 
                     { $chomper = 1;        # turn on!
                       $_ = "$1$2";
                       print FILELOG "(chomp) ";
                     };

                  if (/^(.*)sectionTitle=\`(.+?)\'(.*)$/) 
                     { $SECTIONTITLE = $2;
                       $_ = "$1$3";
                       print FILELOG "title<$SECTIONTITLE> ";
                     };

                  if ( /^(.*)noSections=\`(.+?)\'(.*)$/ )
                     { print FILELOG "nosections=$2 ";
                     if ($2 eq 'yes')
                        { $nosections=1;
                          $_ = "$1$3"; 
                        }
                     elsif ($2 eq 'no')
                        { $nosections=0;
                          $_ = "$1$3"; 
                        };             # if neither yes nor no, ignore!
                     };     
                  if ( /(\w+\s*=\s*\`.+\')/ )  # implies unknown command
                     { &Caution("*Warning* LINE $LINECOUNT " 
                              . "Unknown/duplicated "
                              . "Dogwagger command(s) <$1>");
                     };
\end{verbatim}

The \textsf{\textcolor{blue}{noSections}} test allows you to turn
section headers on or off at will, so we need to accommodate not just
the `yes' option but also the `no' option. Specifying something other
than `yes' or `no' will force an error, as \$\_ is then not updated,
and Section~\ref{sub:An-erroneous-Dogwagger-command} will catch
this below. Although turning on\textcolor{blue}{{} }\textsf{\textcolor{blue}{noSections}}\textcolor{blue}{\small{}
}(disabling printing of section headers) is logically incompatible
with \textsf{\textcolor{blue}{sectionTitle}}\textsf{,} we don't check
for this minor conflict. 




\subsubsection{A new file\label{sub:A-new-file}}

The final legal Dogwagger instruction is\textsf{ }\textsf{\textcolor{blue}{newTarget}}.
The code is here made a bit more complex because I need to accommodate
the possibility that \textsf{\textcolor{blue}{newTarget}} is a uuencoded
file. (I've decreased the indentation, a reflection of the baroque
complexity of WagTheDog). 

%
% DogWagger sectionTitle=` WagTheDog: a new file (Section $[SECTION])'
\begin{verbatim}
                  } else # IS newTarget
                  { 
##################deeply indented section#######################
$RETAINTARGETLINE = "$1$3";      # keep all other specifications
$DOGFILE = $2;                   # get name of new file
# the following line bumps file count _unless_ previously opened 
$cFILES++ unless exists $APPENDTOFILE{ $DOGFILE }; # v4.0.4 (tme)
if ($DOGFILE =~ /^uudecode$/) 
     # if uudecoding do NOT terminate current file!
   { my ($ufile, $umode, $uout) = Uudecode();
     if (length $ufile > 0)
        { &PrintLogLine ("uudecoding <$ufile> mode $umode");
          my ($E2) = 0;
          if (-e $ufile) { $cOUST++; }; # bump overwrite count!
          open UFILE, ">$ufile" 
               or $E2 = &GlobalError("Uudecode failed <$ufile>");
          if (! $E2)
             { binmode UFILE;     # NB otherwise MSDOS stuffup!
               print UFILE $uout; # IGNORE UNIX mode in $umode.
               close UFILE;
        };   
   };
\end{verbatim}

The above code handles the case where the target name is precisely
`uudecode', decoding and writing the uuencoded file (and increasing
the overwrite count \$cOUST if the file exists). Otherwise close the
current target file (Section~\ref{sub:Close-target-file}), read
new target parameters (ReadTargetParams, Section~\ref{sub:Read-target-file-parameters}),
and open a new file (See Section~\ref{sub:Open-the-target}).

%
% DogWagger sectionTitle=` WagTheDog: close file (Section $[SECTION])'
\begin{verbatim}
    } else                        # close current, open new!
    { &CloseDogFile($startComment, $eft, $endComment, $notail);
      # close with old parameters; next, read new...
      ($startComment, $nowarn, $MANDATORY, $sft, $eft, 
       $endComment, $nohead, $nosections, $notail) =
      &ReadTargetParams($RETAINTARGETLINE, $startComment, 
                        $endComment);
      print FILELOG ("\n  Comment format now \"$startComment"
                     . "foo$endComment\"");
      if (! OpenTargetFile($DOGFILE, $startComment, $fido, 
            $nowarn, $sft, $endComment, $nohead)) 
              # nohead.. added 2011-12-18
         { return ("Could not open target: <$DOGFILE>");  #fail
         }; 
    }; 
#################end deeply indented section####################
\end{verbatim}

%
% DogWagger sectionTitle=` WagTheDog: bad command (Section $[SECTION])'
 \begin{verbatim}
                  }; # END of else (is newTarget)
               };    # end of "dogs are allowed"
            };         # end of IS DogWagger ***
\end{verbatim}

Finally, if there are dependencies, we store these (StoreChild, Section~\ref{sub:Store-array-of-lines}).

%
% DogWagger sectionTitle=` WagTheDog: store child (Section $[SECTION])'
 \begin{verbatim}
         if (length $depOn > 0)  # if dependency
            { 
            if (! &StoreChild ($myNam, $depOn, $chomper)) # keep whole
               { &Caution("WARNING: \
                 Input file <$fido> terminated unexpectedly!");
                 close FIDO;
                 close DOGFILE;
                 return ('Sudden INPUT failure'); #fail!
               };
              $myNam = ''; # cannot YET resolve (is a dependency).
            }; 
         };  # end else ... not begin verbatim. 
\end{verbatim}

\newpage{}


\subsection{Hot code}

The preceding code dealt with the case where were are not `hot'. If
we \emph{are} busy writing lines to output (are hot) we next check
for the end of a \verb+verbatim+ statement.

%
% DogWagger sectionTitle=` WagTheDog: hot code (Section $[SECTION])'
 \begin{verbatim}
      } else # ARE HOT (ARE WRITING):
      { 
      if ( /(.*)\\end\{verbatim\}/ ) # end verbatim?
         { 
         if ($OPTN) # OPTION still on?
            { &Alert ( "Optional text not closed. See log!");
              &GlobalError("\n ERROR line $LINECOUNT: "
                         . "+OPTIONAL not closed");
              $OPTN = 0;
            };
 
           $hotdata = $1;
           print DOGFILE $hotdata; # last chunk.
           print FILELOG " ... $LINECOUNT.";
         if (length $myNam > 0)  # if name defined
            { $SECTION = &FixName($myNam, $startComment, 
                                  $SECTION, $endComment, 
                                  $nosections, $SECTIONTITLE); 
            };
           $ishot = 0;             # turn off.
           $chomper = 0;           # back to default
           $chomped = 0;           # redundant.
           $SECTIONTITLE = ''; 
         } else  # NOT end verbatim.. 
\end{verbatim}

If it is the end of a verbatim statement, we print the last bit of
text just before the end. Then, if \verb+myNam+ contains a name,
we resolve all of the dependencies on that name using the function
FixName (Section~\ref{sub:Resolve-labelling-dependencies}). 

Otherwise (it's not the end of a verbatim statement) we check for
the case where we are `chomping' lines (the \textsf{\textcolor{blue}{oneLine}}
option forces multiple verbatim lines to be concatenated, removing
all newline characters).

%
% DogWagger sectionTitle=` WagTheDog: chomping? (Section $[SECTION])'
 \begin{verbatim}
         { 
         if ($chomped)           # Already chomped?
            { / *(.*)/;          # remove leading spaces
              $_ = $1;           # (even allow null line)
            };
         if ($chomper)           # IF line must be chomped
            { print FILELOG '+'; # Amended 2011-12-18
              $chomped = 1;      # signal we've just chomped.
            } else               # UNLESS chomping,
            { $_ = "$_$NEWLINE"; # restore a default newline!
            };  
\end{verbatim}

We also check for +OPTIONAL code, only written if we specified \verb+include=`everything'+

%
% DogWagger sectionTitle=` WagTheDog: OPTIONAL? (Section $[SECTION])'
 \begin{verbatim}
         if ( /^\s*\+OPTIONAL(.*)$/)
            { $OPTN = 1;
              $_ = "";
            if (length $1 > 0)
               { &Caution("Warning LINE $LINECOUNT. "
                        . "Extra +OPTIONAL text");
               };
            };
         if ( /^\s*-OPTIONAL(.+)$/)
            { $OPTN = 0;
              $_ = "";
            if (length $1 > 0)
               { &Caution("Warning LINE $LINECOUNT. "
                        . "Extra-OPTIONAL text");
               };
            }; 

         if ($MANDATORY || ! $OPTN) # unless optional is active
            { print DOGFILE $_; # write to output
            };

         }; # end else not end verbatim
      };   # end else are hot
\end{verbatim}


\subsection{Finish off}

We finish off our enormous \verb+while+ statement, and close the
source file.

%
% DogWagger sectionTitle=` WagTheDog: done (Section $[SECTION])'
 \begin{verbatim}
  }; # end of enormous while stmt.

}  # end of WagTheDog
\end{verbatim}

\newpage{} 


\section{Reading the header\label{sec:Reading-the-header}}

As discussed in the introductory section, we must accommodate the
various header line options. We obtain the version, fileTarget and
newLine values in the following routine:

%
% DogWagger sectionTitle=` Read a header (Section $[SECTION])'
 \begin{verbatim}
sub ReadHeader
{ my ($hotline);
     ($hotline) = @_;
  my ($ver, $target, $comment, $nowarn, $mandatory,
      $sft, $eft, $endComment,
      $nohead, $nosections, $notail);      # $nohead 2011-12-18
  $ver = 0;
  $target = '';
  $comment = '#';                          # aka startComment
  $nowarn = 0;
  $mandatory = 0;
  $sft = '';
  $eft = '';
  $endComment = '';
  $nohead = 0;                              # default write head
  $nosections = 0;                          # and section 
  $notail = 0;
  if ( $hotline !~ /^(.*)version=\`(\d+\.\d+\.\d+)\'(.*)$/ ) 
     { &Caution ("Missing/defective version number");
       return (0,0,0,0, 0,0,0,0, 0,0,0);    #fail
     };
  $ver = $2;
  $hotline = "$1$3";    
  if ( $hotline !~ /^(.*)fileTarget=\`(.+?)\'(.*)$/ )
     { &Caution ("Missing/defective target filename");
       return (0,0,0,0, 0,0,0,0, 0,0,0);    #fail
     };
  $target = $2;
  $hotline = "$1$3";    
  if ($hotline =~ /^(.*)newLine=\`(.+?)\'(.*)$/ )
     { $NEWLINE = $2;                         
       $hotline = "$1$3";    
       $NEWLINE =~ s/\\n/\n/g;              #
       $NEWLINE =~ s/\\r/\r/g;              # replace \n =r
     };
  ($comment, $nowarn, $mandatory, $sft, $eft,
   $endComment, $nohead, $nosections, $notail) =
       &ReadTargetParams($hotline, $comment, $endComment);
  return ($ver, $target, $comment, $nowarn, $mandatory,
          $sft, $eft, $endComment,
          $nohead, $nosections, $notail);  }
\end{verbatim}

Other parameters are obtained by ReadTargetParams (§\ref{sub:Read-target-file-parameters}).


\subsection{Read target file parameters\label{sub:Read-target-file-parameters}}

We read in parameters that are directly related to file creation.
These parameters do \emph{not }include the version number and the\textsf{
}\textsf{\textcolor{blue}{fileTarget}} parameter. In version 2.1
we added the option to specify the very first few characters of the
file using the \emph{startFile} option. This is useful for HTML and
PHP. See also the corresponding\textsf{ }\textsf{\textcolor{blue}{endFile}}
option for terminating the last chunk of a file.

%
% DogWagger sectionTitle=` Read target parameters (Section $[SECTION])'
 \begin{verbatim}
sub ReadTargetParams
{ my ($topline, $comment, $endComment);
     ($topline, $comment, $endComment) = @_;
  my ($nowarn, $mandatory, $sft, $eft, 
      $nohead, $nosections, $notail);  # $nohead.. added 2011-12-18
  my ($alterSC, $alterEC);
  $nowarn    = 0;
  $mandatory = 0;
  $sft       = '';
  $eft       = '';
  $nohead    = 0;           # default write head
  $nosections= 0;           # and section 
  $notail    = 0;
  $alterSC   = 0;           # ON if alter startComment
  $alterEC   = 0;           # on if alter endComment

  if ($topline =~ /^(.*)include=\`everything\'(.*)$/)
     { $mandatory = 1; 
       $topline = "$1$2";
       print FILELOG "include ALL "; };

  if ($topline =~ /^(.*)startComment=\`(.+?)\'(.*)$/) 
     { $comment = $2;       # new startComment
       $topline = "$1$3";
       print FILELOG "comment '$comment'";
       $alterSC = 1;
     };
  if ($topline =~ /^(.*)newComment=\`(.+?)\'(.*)$/) # obsolete.
     { &Caution(  "Minor warning: newComment is deprecated "
                . "(LINE $LINECOUNT). Use startComment." );
       $comment = $2;
       $topline = "$1$3";
       print FILELOG "comment '$comment'";
       $alterSC = 1;
     };
  if ($topline =~ /^(.*)endComment=\`(.*?)\'(.*)$/)
     { $endComment = $2;
       $topline = "$1$3"; 
       print FILELOG "endComment '$endComment'";
       $alterEC = 1;
     };
\end{verbatim}

Because the comment format is carried over, there's a potential problem
if an old format has an \textsf{\textcolor{blue}{endComment }}(e.g.
\verb+-->+) and we change the \textsf{\textcolor{blue}{startComment}}
alone. We thus \emph{reset}\textsf{ }\textsf{\textcolor{blue}{endComment}}
if we've altered\textsf{ }\textsf{\textcolor{blue}{startComment}}
but not \textsf{\textcolor{blue}{endComment}}:

%
% DogWagger sectionTitle=` start v end comment (Section $[SECTION])'
 \begin{verbatim}
  if ($alterSC && ! $alterEC) # if only alter startComment
     { $endComment = '';      # force endComment to default.
     };
\end{verbatim}

The remaining options are straightforward:

% DogWagger sectionTitle=` headline options (Section $[SECTION])'
 \begin{verbatim}
  if ($topline =~ /^(.*)noWarn=\`yes\'(.*)$/)
     { $nowarn = 1; 
       $topline = "$1$2";
       print FILELOG "warn is OFF; "  };
  if ($topline =~ /^(.*)startFile=\`(.*?)\'(.*)$/) # ver 2.1
     { $sft = $2;
       $topline = "$1$3";
       print FILELOG "start code {$sft}; ";
       $sft =~ s/\\n/$NEWLINE/mg; # CR's !!
     };
  if ($topline =~ /^(.*)endFile=\`(.*?)\'(.*)$/)   # ver 2.1
     { $eft = $2;
       $topline = "$1$3";
       print FILELOG "end code {$eft}; ";
       $eft =~ s/\\n/$NEWLINE/mg; # CR's !!
     };
  if ($topline =~ /^(.*)noHeader=\`yes\'(.*)$/ )   # 2011-12-18
     { $nohead=1;
       $topline = "$1$2";
       print FILELOG "NO header; " };
  if ($topline =~ /^(.*)noSections=\`yes\'(.*)$/ )
     { $nosections=1;
       $topline = "$1$2";
       print FILELOG "NO sections; "  };
  if ($topline =~ /^(.*)noTail=\`yes\'(.*)$/ )
     { $notail=1;
       $topline = "$1$2";
       print FILELOG "NO tail; "  };

  if ($topline =~ /(\w+\s*=\s*\`.+\')/ )  # v 4.0.3
     { &Caution("*Warning* LINE $LINECOUNT Unknown/duplicated "
                . "Dogwagger NEWFILE command(s) <$1>");
     };

  return ($comment, $nowarn, $mandatory, $sft, $eft, 
          $endComment, $nohead, $nosections, $notail);  }
\end{verbatim}

\newpage{} 


\section{Miscellaneous routines}

The following are trivial:


\subsection{Confirm an action}

Get console input (confirmation, i.e. `y'):

%
% DogWagger sectionTitle=` Confirm (Section $[SECTION])'
 \begin{verbatim}
sub Confirm
{ my ($msg);
     ($msg) = @_;    

     print "\n$msg";
     my($ans);
     $ans = <STDIN>; # get stdin
     if ($ans =~ /^y/i )
        { return(1);
        };
    return (0);  }
\end{verbatim}


\subsection{Alert}

Alert the user with a warning. Do not alter \$\_.

%
% DogWagger sectionTitle=` Alert (Section $[SECTION])'
 \begin{verbatim}
sub Alert
{ my ($msg);
     ($msg) = @_;     
     print "$msg";  }
\end{verbatim}


\subsection{ChompLine\label{sub:ChompLine}}

Remove CR/LF characters from the start/end of a line:

%
%DogWagger sectionTitle=` Chomp line (Section $[SECTION])'
 \begin{verbatim}
sub ChompLine
{ my ($line);
     ($line) = @_;
  if ( $line =~ /([^\n\r]*)/ms )
                { $line = $1;
                };
  return($line);  }
\end{verbatim}


\subsection{Determine newline character(s)}

Given a sample line, find out the codes used eg 0D or 0D0A. Unescape
is below (§\ref{sub:Unescape}).

%
%DogWagger sectionTitle=` Get newline characters (Section $[SECTION])'
 \begin{verbatim}
sub GetNewLine
{ my ($line);
     ($line) = @_;
  if ( $line =~ /[^\n\r]*([\n\r]+)$/ms )
                { my($nl) = $1;
                  $_ = &Unescape($1);
                  print FILELOG "\n  newline is <$_> ";
                  return($nl);
                };
  print FILELOG "\n  Resorting to newline default.";
  return($DEFAULTNEWLINE); }
\end{verbatim}


\subsection{Caution --- Alert with print}

`Caution' always generates an alert. This code should not alter \$\_.

%
% DogWagger sectionTitle=` Caution (Section $[SECTION])'
 \begin{verbatim}
sub Caution 
{ my ($msg);
     ($msg) = @_;
  print FILELOG "\n$msg";  # must prepend newline.
  &Alert("\n$msg"); }
\end{verbatim}


\subsection{Read the time\label{sub:Read-the-time}}

This is the local machine time.

%
% DogWagger sectionTitle=` Get local time (Section $[SECTION])'
 \begin{verbatim}
sub GetLocalTime
{  my ($sec, $min, $hour, $mday, $mon, 
       $year, $wday, $yday, $isdst);
($sec, $min, $hour, $mday, $mon, 
       $year, $wday, $yday, $isdst) = localtime(time);

  $year += 1900;      #fix y2k.
  $mon ++;            #january is zero!
  return ("$year-$mon-$mday $hour:$min:$sec"); }
\end{verbatim}


\subsection{An error-related hack}

Clumsy.

%
% DogWagger sectionTitle=` global error (Section $[SECTION])'
 \begin{verbatim}
sub GlobalError
{ my ($msg);
     ($msg) = @_;
  print FILELOG "\n$msg";     
  $ERRCOUNT ++; # bump error count
  return($msg);
}
\end{verbatim}


\subsection{Pretty up codes\label{sub:Unescape}}

We look for \textbackslash{}r and \textbackslash{}n (carriage return,
line feed) and substitute text representations:

%
% DogWagger sectionTitle=` Unescape (Section $[SECTION])'
 \begin{verbatim}
sub Unescape
{ my ($t);
     ($t) = @_;
  $t =~ s/\n/\\n/ms;
  $t =~ s/\r/\\r/ms;
  return($t);
}
\end{verbatim}

\newpage{}


\section{Deferred code}

The following routines allow you to write code that is deferred until
the code it depends on has been resolved. Do not assume that if A
and B both depend on C, that A or B will be resolved in any particular
order, unless you explicitly state that A depends on B, or vice versa!


\subsection{Store array of lines\label{sub:Store-array-of-lines}}

Here we keep `child' (dependent) lines in an array, to be resolved
later when all dependencies are met. The index of the topmost child
is given by \verb+$#CHILDREN+. The list of dependencies contains
elements separated by commas, and there are starting and terminal
commas to facilitate matching.

%
% DogWagger sectionTitle=` Store child (Section $[SECTION])'
 \begin{verbatim}
sub StoreChild 
{  my ($pendingName, $dependencies, $chomper);
      ($pendingName, $dependencies, $chomper) = @_;
   my ($idx, $child);
   $idx = 1+$#CHILDREN;
   print FILELOG "Keeping child[$idx].";
   
   $_ = <FIDO>; # first line *must* be begin verbatim
                # what if this is not defined?? (check me)
   $LINECOUNT ++;
   $_ = &ChompLine($_); 
   if ($chomper)           # IF line must be chomped
      { print FILELOG '/'; # 
      } else               # UNLESS chomping,
      { $_ = "$_$NEWLINE"; # restore a default newline!
      };  
\end{verbatim}

The above allows a\textcolor{blue}{{} }\textsf{\textcolor{blue}{oneLine=\textrm{\large`}true'}}\textcolor{blue}{{}
}statement to affect a deferred block. We now make sure that the line
begins with a new \verb+verbatim+ statement.

%
% DogWagger sectionTitle=` ensure verbatim (Section $[SECTION])'
 \begin{verbatim}
   if ( /\\begin\{verbatim\}(.*)/ )
      { $child = $1;  # keep rest of line
      } else
      { print FILELOG "\n*ERROR at line $LINECOUNT:"
         . " no verbatim stmt on 1st child line!";
        $ERRCOUNT ++; # bump error 
        print FILELOG "<$ERRCOUNT!>";
        print FILELOG "<$_>";
        return 1; # not fatal, per se. 
      };
\end{verbatim}

Next, store the name of the pending section, and its dependencies,
all at the top of the current (global) lists. The actual text is stored
in \verb+$child+. See how the dependencies list has a leading comma.

%
% DogWagger sectionTitle=` dependencies (Section $[SECTION])'
 \begin{verbatim} 
   $DEPENDENCIES[$idx] = ",$dependencies,"; 
   $PENDINGNAME[$idx] = $pendingName;
   $CHILDREN[$idx] = ''; # default nothing
   my($ok) =1;
\end{verbatim}

We sequentially read in lines until end-of-file (an error), or the
verbatim statement ends.

%
% DogWagger sectionTitle=` sequential read (Section $[SECTION])'
 \begin{verbatim} 
   while($ok)
     { $_ = <FIDO>;
       if (! defined)
          { return 0; # fail
          } else
          { $_ = &ChompLine($_); 
            $LINECOUNT ++;
            if ($chomper)             # IF line must be chomped
               { print FILELOG '/';   
               } else                 # UNLESS chomping,
               { $_ = "$_$NEWLINE";   # restore a default newline!
               };  
           if ( /(.*)\\end\{verbatim\}/ )
              { $ok = 0;
                $child = "$child$1";  # append last section
              } else
              { $child = "$child$_";  # concatenate.
              };  
          };
     };
  $CHILDREN[$idx] = $child; # store away lines to be printed   
  return 1; 
} # success!
\end{verbatim}


\subsection{Resolve labelling dependencies\label{sub:Resolve-labelling-dependencies}}

FixName walks through all dependencies, resolves them (where relevant),
and on resolution, writes the relevant child code to output. (A `child'
is a section which depends on other sections, and must not be written
before these sections have been identified and written).

The arguments submitted are the name to be resolved, the codes for
starting and ending a comment (c, ec), whether section headers will
be written (nosections), and the \verb+$SECTION+ number. Note that
the last-mentioned will be amended and returned.

%
% DogWagger sectionTitle=` Fix name (Section $[SECTION])'
 \begin{verbatim}

sub FixName
{ my ($fname, $c, $SECTION, $ec, $nosections, $SECTIONTITLE); 
     ($fname, $c, $SECTION, $ec, $nosections, $SECTIONTITLE) = @_; 
     # nosections added 2011-12-18
  my ($morenames) = ",$fname,"; 
  my ($idx);
\end{verbatim}

We now enter a while loop that takes the `name to be resolved' off
\verb+morenames+ and then enters an inner \emph{while} loop that
examines each pending name to see whether it's dependent on the name
to be resolved.

%
% DogWagger sectionTitle=` fixname while loop (Section $[SECTION])'
 \begin{verbatim}
  while ( $morenames =~ /^(.*,)(.+),$/ ) # split off last name
    { $fname = $2;
      $morenames = $1; 
      $idx = $#CHILDREN;                         
      while ($idx > -1 )                 # for each child entry
        { if ($DEPENDENCIES[$idx] =~ /(.*,)$fname,(.*)/ ) 
             { $_ = "$1$2";              # if name in list, clip
               print FILELOG " (child[$idx] now has <$fname>) ";
               $DEPENDENCIES[$idx] = $_; # rewrite
               if ( /^,$/ )              # if all resolved 
                  {  print FILELOG "WRITING child[$idx] ";
                     $SECTION = &PrintSectionHeader($c, $SECTION,
                                         $ec, $nosections,
                                         $SECTIONTITLE);
                     print DOGFILE $CHILDREN[$idx];
                     $CHILDREN[$idx] = ''; 
\end{verbatim}

Once we've resolved this dependency, we realise that other files may
in turn be dependent on the child that has been resolved. We get the
name of \emph{this} child off \verb+@PENDINGNAME+ and add it to \verb+$morenames+.

%
% DogWagger sectionTitle=` and repeat (Section $[SECTION])'
 \begin{verbatim}
                     # ....WAIT! this child may have dependencies!
                     if (length $PENDINGNAME[$idx] > 0) # if so ...
                        { $morenames = 
                             "$morenames$PENDINGNAME[$idx],";
                             # add name of resolved child!
                          $PENDINGNAME[$idx] = '';# clear me!
             };   };    };
          $idx --;                       # move to preceding child
        };                               # all children done.
    };
  return $SECTION; }
\end{verbatim}

The above explains why we \emph{must} work backwards through \verb+@DEPENDENCIES+.
Earlier entries may depend on later ones, but later ones will have
been resolved at entry. It's therefore wrong to start \verb+$idx+
at zero and count up.


\subsection{Check for unresolved dependencies}

At the end, we have to make sure that all dependencies have been resolved,
or signal an error. Errors are also written to the log.

%
% DogWagger sectionTitle=` Check unresolved dependencies (Section $[SECTION])'
 \begin{verbatim}
sub CheckUnresolved
{ my($idx);
  $idx = $#CHILDREN;
  my ($errcnt);
  $errcnt = 0;  
  while ($idx > -1)
    {
      if (length $CHILDREN[$idx] > 0)
         { print FILELOG "\n\n *** ERROR *** "
                         . "\n\n Unresolved code: \n ";
           print FILELOG "Dependencies: <$DEPENDENCIES[$idx]> \n";
           print FILELOG "Name: <$PENDINGNAME[$idx]> \n"
                         . " <Code: <---\n ";
           print FILELOG $CHILDREN[$idx];
           print FILELOG "$\n ---> Code ends> \n\n";
           $errcnt++;
         };
      $idx --;
    };
  return $errcnt; } # number of errors, 0=ok.
\end{verbatim}

\newpage{} 


\section{Handle multiple files}

This section is a consequence of Dogwagger's ability to generate
multiple files from a single .TEX source. We chose the simple option
of closing the first file and then opening and writing the next one,
rather than having multiple dangling file handles. The sole exception
to this rule is that if we are writing a UUdecoded binary file, we
don't fiddle with the currently open file.


\subsection{Open the target\label{sub:Open-the-target}}

We open a target file, warning appropriately about overwrites.

%
% DogWagger sectionTitle=` open target file (Section $[SECTION])'
 \begin{verbatim}
sub OpenTargetFile
{ my ($DOGFILE, $c, $fido, $nowarn, $sft, $ec, 
      $nohead);  # nohead added 2011-12-18
     ($DOGFILE, $c, $fido, $nowarn, $sft, $ec,
      $nohead) = @_;
     my($ok);
  $TODAY = &GetLocalTime();
\end{verbatim}

The following code sets the prefix to overwrite if the filepath does
not occur in the hash \textbackslash{}verb+\$APPENDTOFILE+, otherwise
to append.

%
% DogWagger sectionTitle=` decide to overwrite or append (Section $[SECTION])'
\begin{verbatim}
  my $prefix = ">";  #  start amendment for v4.0.4 (tme)
  $prefix = ">>" if exists $APPENDTOFILE{ $DOGFILE };
  $APPENDTOFILE{ $DOGFILE } = 1;  # end amendment v4.0.4 (tme)
\end{verbatim}

We check that the file exists, and warn if overwriting, unless this
warning has been suppressed:

%
% DogWagger sectionTitle=` warn if overwriting (Section $[SECTION])'
 \begin{verbatim}
  if ($prefix eq ">" and -e $DOGFILE) # if prefix condition added v4.0.4 (tme)
     { if (! $nowarn)  # and warning enabled
          { if (! &Confirm ( 
                     "Overwrite <$DOGFILE>? Are you sure?"))
               { &Caution( "[NOT overwriting $DOGFILE]");
                 $DOGFILE = 'JUNK.JUNK'; # write to junk file.
               } else
               { print FILELOG "[Overwriting $DOGFILE]";
                 $cOUST++;  # increase overwrite count.
               };
           } else
           { $cOUST++;  # increase overwrite count.
     };    };
  my ($E3) = 0;
\end{verbatim}

If not overwriting, rather than simply discarding the text, we rather
clumsily write to the file \emph{JUNK.JUNK}. 

%
% DogWagger sectionTitle=` discard file (Section $[SECTION])'
 \begin{verbatim}
  open DOGFILE, $prefix."$DOGFILE" or  # ">" changed to $prefix v4.0.4 (tme)
        $E3 = &GlobalError(
                "Could not open target <$DOGFILE> :$!");
     if ($E3)
        { return 0; 
        }; #fail
  print FILELOG "\n\n==Opened target file <$DOGFILE> ";
\end{verbatim}

On opening the target file, we write several lines to this file (\verb+DOGFILE+)
using the comment character(s) to rem out the lines.

%
% DogWagger sectionTitle=` write headlines (Section $[SECTION])'
 \begin{verbatim}
  print DOGFILE $sft; # very first text eg. for PHP.
                      # print EVEN IF noHead=`yes'.
  if (! $nohead)      # added 2011-12-18
  { print DOGFILE 
      "$c Generated by LaTeX DogWagger Version " . 
      "$MAJORVERSION.$MINORVERSION.$TV "
      . "from file <$fido>$ec$NEWLINE";
    print DOGFILE "$c Date: [$TODAY] $ec$NEWLINE";
    print DOGFILE "$c Do NOT edit this file. "
            . "Edit the LaTeX source!!$ec$NEWLINE"; 
                      # \n appended 2011-12-18. Is this wise?
  };
  return 1; # success
}
\end{verbatim}

The \textsf{\textcolor{blue}{noHead=`yes'}}\textcolor{blue}{{} }option
allows us to suppress the Dogwagger header text, but if we specified
\textsf{\textcolor{blue}{startFile}} text, it is still (of course)
written. 

\newpage{}


\subsection{Close target file\label{sub:Close-target-file}}

Simply close the output file handle \verb+DOGFILE+.

%
% DogWagger sectionTitle=` Close target file (Section $[SECTION])'
 \begin{verbatim}
sub CloseDogFile
{ my ($c, $eft, $ec, $notail);
     ($c, $eft, $ec, $notail) = @_;
  my ($EC) = &CheckUnresolved();
  $ERRCOUNT += $EC;      # global
  if ($EC > 0)
     { print FILELOG "\nUnresolved errors <$EC>";
       if (! $notail)
          { print DOGFILE  
              "$NEWLINE$NEWLINE$c -- WARNING:"
              . " $EC errors. See log!$ec$NEWLINE";
     };   };    
\end{verbatim}


We can suppress writing of terminal (`tail') Dogwagger code, but the
final code in \verb+$eft+ is written (if it exists) regardless of
the value in \verb+$notail+.

%
% DogWagger sectionTitle=` write tail (Section $[SECTION])'
 \begin{verbatim}
  if (! $notail)
     { print DOGFILE "$c -END OF FILE- $ec$NEWLINE";
     };
     print DOGFILE $eft; # Print EVEN if noTail=`yes'
     close DOGFILE;      
     print FILELOG " FILE CLOSED== ";  
}
\end{verbatim}

\newpage{} 


\section{Trivial amendments}


\subsection{Print a section header\label{sub:Print-a-section-header}}

We now have the ability to label sections with pre-defined text (commented
out). Remember that \verb+$LINECOUNT+ is an ugly global. We submit
the section count \verb+$SECTION+, and return this value incremented
by one.

%
% DogWagger sectionTitle=` Print section header (Section $[SECTION])'
 \begin{verbatim}
sub PrintSectionHeader
{ my($c, $SECTION, $ec, $nosections, $SECTIONTITLE); 
     # $nosections 2011-12-18
    ($c, $SECTION, $ec, $nosections, $SECTIONTITLE) = @_;
  if (! $nosections)
  { if (length $SECTIONTITLE > 0)
       { $_ = $SECTIONTITLE;
         if (/\$\[SECTION\]/) # if contains section count
            { s/\$\[SECTION\]/$SECTION/;
            };
            print DOGFILE "$NEWLINE$c$_$ec$NEWLINE";
       } else
       { print DOGFILE "$NEWLINE$c - <Section $SECTION>"
                             . " - $ec$NEWLINE";
  };   };
  &PrintLogLine ("writing section [$SECTION]");
  $SECTION ++;
  return $SECTION;  
}
\end{verbatim}

The above code also replaces the text \textsf{\textcolor{blue}{\$[SECTION]}}
(within the section header) with the actual section count. 


\subsection{Print LOG line}

We often print to the log with a line number, so let's formalise this:

%
% DogWagger sectionTitle=` Print log line (Section $[SECTION])'
 \begin{verbatim}
sub PrintLogLine
{ my($t); 
   ($t) = @_;
   print FILELOG "\n  line $LINECOUNT: $t"; 
}
\end{verbatim}

It makes code slightly more concise, at little cost. 

\newpage{} 


\section{Binary encoding and decoding}

Responding to the need to write binary code from our TEX source:


\subsection{UUdecode}

We read the global file handle FIDO, mandating that the initial line
is the `begin verbatim' line. The line immediately after this must
contain the UUencoded header line. The subsidiary routine UUdecodeLine
returns not only decoded text, but also an error code. If the error
code is less than zero, an error has occurred; if the error code is
zero, then the subsequent line \emph{must} be an \textcolor{green}{end}
statement signalling the end of the UUencoded section!

%
% DogWagger sectionTitle=` Uudecode (Section $[SECTION])'
 \begin{verbatim}
sub Uudecode  
{ my ($filename, $mode, @rslt);
  my ($line, $decoded, $err);
  $filename = "";
  $line = <FIDO>; # this should be \begin{verbatim} line:
  $LINECOUNT ++;
  if ($line !~ /\\begin\{verbatim\}/ )
     { &Alert ("Uudecode: no verbatim <$line>");
       return ("", "", "");
     };

  $line = <FIDO>; # MUST be header!
  $LINECOUNT ++;
  $line = &ChompLine($line); 
  $line =~ /begin\s+(\d{3})\s+(.+)/; 
  if (! defined $1)
     { # here write error!
       &Alert ("Uudecode: bad first UU line <$line>");
       return ("", "", "");
     };
  $mode = $1;
  $filename=$2;
  $err = 1; # -ve will signal failure
  
  while ( $line = <FIDO> ) 
    { # hmm what if extra 0xD ?
      last if (! defined $line); # ??
      $LINECOUNT ++;
      $line = &ChompLine($line); 
      last if ($line =~ /^end/);
      if (! $err) # bad if err zero
         { &Alert ("Uudecode: end stmt not seen!<$line>");
           last;
         };
      ($decoded, $err) = UudecodeLine($line);
      # nb if $err is zero, next line must be /^end/!
      if ($err < 0)
         { # here could write error!
           if ($err == -1)
              { $err = "Bad line";
              } 
           elsif ($err == -2)
              { $err = "silly length($decoded)";
              }
           elsif ($err == -3)
              { $err = "lengths don't match($decoded)";
              };
              
           &Alert ("Uudecode: error $err in <$line>");
           last; # terminate
         };
      push @rslt, $decoded;
    };
  return ($filename, $mode, join("",@rslt));
}
\end{verbatim}


\subsection{UUdecode line}

Here's the routine that actually does the business of UUdecoding.
We've kept this very simple, based on publicly available code.%
\footnote{\href{ http://search.cpan.org/src/ANDK/Convert-UU-0.52/lib/Convert/UU.pm}{ http://search.cpan.org/src/ANDK/Convert-UU-0.52/lib/Convert/UU.pm}%
} On most UNIX/Linux systems, UUencoding and decoding should be readily
available, but for DOS uuencoding, try e.g. \href{http://drn.digitalriver.com/product.php\%5Bid\%5D39753\%5Bcid\%5D2\%5BSiteID\%5Ddriverguide}{this program.}

%
% DogWagger sectionTitle=` Uudecode line (Section $[SECTION])'
 \begin{verbatim}
sub UudecodeLine
{ my ($line) = @_;
  my ($charlen);
  my ($decoded, $ld);
  
  $line =~ /(.).*\`*$/; # remove terminal backticks too!
  if (! defined $1)
     { return ("", -1); # dud line!
     };
   $charlen = (ord($1) - 32) & 077;
  if ($charlen == 0)
     { # ie terminal line with single backtick:
       # no error, but END!
       return ("", 0);
     };
  if (($charlen > 45) || ($charlen <0))
     { return ("$charlen($1)", -2); # bad length
     };

 # convert to number, then count of characters encoded;
  $decoded = unpack("u", $line); #uudecode!
  $ld = length $decoded;
  if ($ld != $charlen)
     { return ("$ld:$charlen:$decoded", -3); # length doesn't match!
     };
  return ($decoded, 1); # success!
}
\end{verbatim}\vskip -2mm

See the archaic use of octal. But it works.


\subsubsection*{A brief note on uuencoding/decoding}

% uuencoding rules:
% http://en.wikipedia.org/wiki/UUcoding
% http://www.gsp.com/cgi-bin/man.cgi?section=5&topic=uuencode


This description assumes you understand hexadecimal and ASCII. A uuencoded
file consists of: 
\begin{enumerate}
\item A header line; 
\item A body; 
\item A trailer line. 
\end{enumerate}
All other lines must be ignored Lines may end with 0x0D, 0x0A, or
any combination of the two (ie carriage return and/or line feed).
From now on we'll call the end of line character(s) the 'endline'.
Conventionally this should be \emph{encoded} as simply 0x0A. Here
are the details:
\begin{enumerate}
\item The header line. This contains three items \emph{separated} by spaces
(0x20): 

\begin{enumerate}
\item The five character string 'begin' (no quotes around it) 
\item Three digits, each in the range 0..7 i.e. an octal number 
\item The file name in ASCII (potential for trouble here!) 
\end{enumerate}

The header line terminates with an endline.

\item The body. This contains one or more lines, each ending with an endline.
For all but the last data line, there should be 62 characters in a
line: 

\begin{enumerate}
\item A single character, usually the ASCII character M 
\item 60 characters representing an encoded string 
\item The endline 
\end{enumerate}

For the last line, some variation is permitted: The first character
can be in the range 0x20 to 0x5F. There's a FURTHER CHECK: the very
last line of the body does NOT contain data and is simply made up
of a single backtick character.


\emph{In all cases}: 
\begin{enumerate}
\item The first character represents the number of encoded characters, with
0x20 added! This is why all lines but the last should start with M:
they contain 45 encoded characters (hex 0x3D). The ASCII representation
of M is 5D, ie 0x3D + 0x20. 
\item Encoding of characters is done three-at-a-time. If there are less
than three characters, we pad with hex zero (0x0). Encoding is as
follows: 

\begin{enumerate}
\item Divide the 3 bytes (3{*}8 = 24 bits) into four groups of 6 bits, working
from left to right; 
\item This gives us four numbers between 0x0 and 0x3F; 
\item To each number, add 0x20, giving numbers in the range of 0x20--0x5F; 
\item Write the numbers as four ASCII characters to the output file. 
\end{enumerate}

In other words, we output characters in the set of:

\end{enumerate}

 \verb+!"#$%&'()*,-./:;<=>?@[\]^_+ as well as plus, space, 0--9 and
A--Z.

\item The trailer line. This starts with the three character string `end'
(No quotes).
\end{enumerate}
There are some frills: 
\begin{itemize}
\item Also permissible are ASCII characters \$>\$ 95 (5Fh) but only the
rightmost 6 bits are relevant.
\item The three digit number on the first line is the file mode (read /
write / execute) The first number is the octal representation of the
read permission of the file, the next the write permission, and finally
the execute permission.
\item Because some mailers used to strip off terminal blanks, it is usual
(and perhaps wise) to pad such lines (with at least one terminal blank)
with supplementary junk characters. The usual character is the backtick,
0x60, which has the added advantage that it translates to the otherwise
illegal value 0x0 when the high bits are masked off.
\end{itemize}
\newpage


\section{Change log}


\subsection{Changes in version 2.0}

The major changes in version 2.0 were related to the ability to shift
sections down below other sections (defer writing of certain sections),
waiting for all dependencies to be fulfilled before writing the code.
Names are only resolved when all of the dependencies of that section
have been met.

It would be possible to keep a record of 'names already resolved'
but this is a little silly. We are only interested in deferring the
writing of code, so there's little point in bookkeeping to this extent.
Just leave out the dependency if it's already resolved! 

If a child has no name, then the corresponding PENDINGNAME element
is~\verb+''+. We could decrease memory requirements for CHILDREN
by deleting array elements once written; we might benefit from even
removing all corresponding elements entirely (as we will resolve the
name immediately and the element in DEPENDENCIES is of course empty
as well).

We also introduced the concepts of line concatenation, chomping off
line feeds and subsequent leading spaces.%
\footnote{When using the WinEdt text editor, the default is to trim trailing
spaces, which can be rather irritating. You have to uncheck Options|Preferences|defaults|Trim
spaces, or in already created documents uncheck Document|document
settings|trim spaces.%
}

Another amendment was allowing alteration of the initial comment string
(formerly newComment, now startComment), the noWarn option to suppress
irritating warnings (especially with multiple file writes), and OPTIONAL
statements for a debugging version. 

We also introduced insertion of binary (uuencoded) files, by allowing
the user to say \verb+newTarget=`uudecode'+. Usage of the Uudecode
function is:

\vskip 1mm

\verb+($filename, $mode, $outstring) = Uudecode();+

\vskip 1mm 




\subsection{Changes in version 2.1}
\begin{enumerate}
\item We allow file start and end code. This is really for PHP, where such
code is vital, but also for included HTML. The directive we use, for
example startFile=`$<$?php' must be in the same line as the file
name specification, but we can specify the endFile='?$>$' directive
any time before we terminate the file for which we require such terminal
code. The length of either can be specified as zero using startFile=`'
or endFile=`'. \textbf{Note} that after each file is written, these
text strings are reset to the null string, so the startFile and endFile
values must be specified for \emph{each file} in which they are used!
\item We removed use of \verb+=cut+, as we had problems with some Perl
versions, notably v5.8.
\item We set the focus to the `Wag' button, so on running the program with
a file argument, things are ready to run.
\item We allowed comment closure (as in HTML, older versions of C), along
the lines of \emph{endComment=`--$>$'} for HTML. The default closing
comment is an empty string. Note that unlike startFile and endFile,
startComment persist until changed. (As of version 4.0.0, applies
to endComment too). 
\item We fixed a problem with multiple dependencies, where a section was
printed after just one dependency was resolved owing to an error in
testing the remaining list of dependencies! 
\item In minor version 2.1.1, if the user declines to overwrite a given
file, we don't abort the whole Dogwagger process --- instead, we write
the data to the file `JUNK.JUNK' and carry on! Note that we overwrite
this file, and don't append, so if you decline to overwrite multiple
files, only the last will be kept. 
\item In minor version 2.1.2, we remove the text \verb+--+ from the end
file and section annotations, to prevent conflict with HTML formatting
checks.
\end{enumerate}

\subsection{Changes in version 3.0}

Okay, basically just got rid of Tk, and rewrote things to accommodate
this.


\subsubsection*{Version 3.0.1}

Fixed error with line counts. Note that in \LyX{}, we can really mess
things up by having a line continue with the ERT containing a verbatim
statement, as this is now missed by DogWagger. 

Also print the name of the file being processed to stdout, and put
in a carriage return before asking about \textquotedbl{}Overwrite\ldots{}?\textquotedbl{}


\subsection{Version 4.0}

A few `issues', sorted out on 2011-12-18 -- 2011-12-19. This started
as a minor rewrite and expanded a bit, so the new version number seems
justified.
\begin{enumerate}
\item Because of issues with chomp, the section that used this to concatenate
multiple lines into one (\verb+oneLine=`yes'+) has been rewritten.
\item Allied to the preceding issue, different source documents will use
different conventions for a new line. Conventions are:

\begin{itemize}
\item Macintosh (obsolete) 0D (\verb+\r+) 
\item Unix/Linux, and Mac OS X: 0A (\verb+\n+) 
\item MS Windows 0D,0A (\verb+\r\n+) 
\end{itemize}

We will generally need to preserve the characteristics of a document
faithfully, but should provide the option of recoding using a different
convention (\verb+newLine=`whatever'+).

\item If we had \verb+noWarn=`yes'+ before any other parameterised statement,
this failed, because we used greedy matching. Fixed.
\item We really do need some mechanism of suppressing the header comments
that are inserted by DogWagger. I thus introduced the option:\\
 \verb+noHeader=`yes'+
\item While we're about it, let's add \verb+noSections+ and \verb+noTail+
too.
\item Because suppression of header, sections and tail is against our basic
philosophy, and we don't want any `surprises', we will reset these
options to the default on closing any written file. The problem is
that up till now, we have read several file-related parameters {}``on
the fly''. These include:

\begin{itemize}
\item startComment / newComment 
\item include 
\item noWarn 
\item endComment 
\item startFile 
\item endFile 
\item noHeader 
\item noTail 
\end{itemize}

We now do the honourable thing, and read these parameters (ReadTargetParams)
when we encounter newTarget! This does raise the possibility that
someone might wish to change these parameters within a file. Hmm.

\item It's nice to have \verb+perl Dogwagger404.pl --help+ produce a list
of options. (done).
\item We now warn about unrecognised options. This includes default options
that are automatically reset and thus should not be specified.
\item Fixed the bug that permits commented-out verbatim statements from
being included.
\item Having newComment is quite confusing, so I deprecate this, and recommend
simply using startComment consistently.
\item The startComment / endComment retention predisposes to an unwanted
endComment being retained after startComment has changed. We should
thus automatically turn the endComment to '' if the startComment
is altered BUT the endComment is NOT! This change has been made in
ReadTargetParams.
\item Line counts now work after deferred code and uudecoding.
\item We need the option to write to a different log file (distinct from
the usual `WAGLOG.LOG'). I have therefore introduced the command line
option of log=logfilename.
\end{enumerate}
I've also prettied up the code to get rid of the nasty text overruns
in the PDF document, and made some minor amendments to neaten up the
text.


\subsubsection*{Version 4.0.1}
\begin{enumerate}
\item Change from \LaTeX{} to \LyX{}, with a few frills (and a logo);
\item A major revision of the documentation;
\item Less picky about case in searching for {}``dogwagger'';
\item Cleaned up checking for `uudecode' string;
\item Introduced ability to arbitrarily turn section headings on and off
(not only at file start);
\item Maintain (and print) counts of number of sections used, files encoded,
etc.
\item Prettied up (and standardised) log output.
\end{enumerate}



\subsubsection*{Version 4.0.2}

A few bugs and amendments.
\begin{enumerate}
\item noSections = `yes' fails in 4.0.1. This is because when I rapidly
added the code to permit suppression/activation of individual sections,
I failed to anticipate the following problem: the command is excised;
section suppression is turned on but then immediately off (by default)
at the start of the file; and the command is no longer visible when
the new file parameters are read. Fix by checking for newTarget in
Section~\ref{sub:Not-hot}. 
\item Section~\ref{sub:Use-Dogwagger-with-LyX} rewritten to accommodate
a message about the best location of the very first Dogwagger line
(in the \LaTeX{} preamble). 
\item Some reformatting. The use of Sans-serif throughout is just too in-your-face,
so I've changed to Roman. The base font size is a bit problematic,
so some of the sections have been rendered as \textbackslash{}footnotesize
to make them fit. 
\item {[}Note that as things stand, uuencoded files cannot be stored to
subdirectories. Should we fix this?{]}
\end{enumerate}

\subsubsection*{Version 4.0.3}

Oops. We really need to rethink the poor code in Section~\ref{sub:Not-hot}.
The silliness of this code reflects the harm that initial bad coding
inflicts on subsequent versions.  
\begin{enumerate}
\item Some reformatting of the offending section;
\item Changed test for {}``unrecognised commands'' to allow for the possibility
of (illegal) whitespace in between \textbf{command} and \textbf{=}
and \textbf{`parameters'}. 
\item Search the first 100 lines for the initial Dogwagger line (\LyX{}
can put in a lot of stuff). 
\end{enumerate}

\subsubsection*{Version 4.0.4}

On 15 May 2012, Mark Ellison insightfully pointed out that some authors
may wish to intersperse file components, specifying parts of \emph{different
files} `out-of-order'.%
\footnote{The program actually does allow you to defer a section until after
the current file closes, but this is clumsy and generates an error
message.%
} Even better, he made small amendments that address the problem. These
are all labelled `v4.0.4 (tme)' in the preceding code:
\begin{enumerate}
\item Re-entrancy is set up in Section~\ref{sub:Setup-for-file-reentrancy}
using an associative array;
\item The file count is only bumped if we are opening a new file (Section~\ref{sub:A-new-file});
\item When we open the target file (Section~\ref{sub:Open-the-target})
we \emph{append} if we've previously opened this file during the current
run.
\end{enumerate}
There are a few caveats. The obvious one is that you will generally
need to specify all of the necessary header parameters every time
you open the file, for example the comment format (unless the files
you are flicking between use the same format). A second and slightly
more ominous `gotcha' concerns \texttt{startFile} and \texttt{endFile}.
The startup code (for e.g. PHP or HTML) should only be specified the
first time you open the file, and the end code should only be specified
the last time you open the file to append code. 

\newpage{}

\appendix
%dummy comment inserted by tex2lyx to ensure that this paragraph is not empty



\section[GNU GPL]{{\large Appendix: GNU GENERAL PUBLIC LICENSE} \protect \\
 {\large Version 2, June 1991} \label{sec:Appendix:-GNU-GPL}}

Copyright (C) 1989, 1991 Free Software Foundation, Inc.\\
 59 Temple Place - Suite 330, Boston, MA 02111-1307, USA \medskip{}


\emph{Everyone is permitted to copy and distribute verbatim copies
of this license document, but changing it is not allowed. }


\subsection*{Preamble}

{\small The licenses for most software are designed to take away your
freedom to share and change it. By contrast, the GNU General Public
License is intended to guarantee your freedom to share and change
free software--to make sure the software is free for all its users.
This General Public License applies to most of the Free Software Foundation's
software and to any other program whose authors commit to using it.
(Some other Free Software Foundation software is covered by the GNU
Library General Public License instead.) You can apply it to your
programs, too. }{\small \par}

{\small When we speak of free software, we are referring to freedom,
not price. Our General Public Licenses are designed to make sure that
you have the freedom to distribute copies of free software (and charge
for this service if you wish), that you receive source code or can
get it if you want it, that you can change the software or use pieces
of it in new free programs; and that you know you can do these things. }{\small \par}

{\small To protect your rights, we need to make restrictions that
forbid anyone to deny you these rights or to ask you to surrender
the rights. These restrictions translate to certain responsibilities
for you if you distribute copies of the software, or if you modify
it. }{\small \par}

{\small For example, if you distribute copies of such a program, whether
gratis or for a fee, you must give the recipients all the rights that
you have. You must make sure that they, too, receive or can get the
source code. And you must show them these terms so they know their
rights. }{\small \par}

{\small We protect your rights with two steps: (1) copyright the software,
and (2) offer you this license which gives you legal permission to
copy, distribute and/or modify the software. }{\small \par}

{\small Also, for each author's protection and ours, we want to make
certain that everyone understands that there is no warranty for this
free software. If the software is modified by someone else and passed
on, we want its recipients to know that what they have is not the
original, so that any problems introduced by others will not reflect
on the original authors' reputations. }{\small \par}

{\small Finally, any free program is threatened constantly by software
patents. We wish to avoid the danger that redistributors of a free
program will individually obtain patent licenses, in effect making
the program proprietary. To prevent this, we have made it clear that
any patent must be licensed for everyone's free use or not licensed
at all. }{\small \par}

{\small The precise terms and conditions for copying, distribution
and modification follow. }{\small \par}


\subsection*{TERMS AND CONDITIONS FOR COPYING, DISTRIBUTION AND MODIFICATION}
\begin{enumerate}
\item This License applies to any program or other work which contains a
notice placed by the copyright holder saying it may be distributed
under the terms of this General Public License. The \textquotedbl{}Program\textquotedbl{},
below, refers to any such program or work, and a \textquotedbl{}work
based on the Program\textquotedbl{} means either the Program or any
derivative work under copyright law: that is to say, a work containing
the Program or a portion of it, either verbatim or with modifications
and/or translated into another language. (Hereinafter, translation
is included without limitation in the term \textquotedbl{}modification\textquotedbl{}.)
Each licensee is addressed as \textquotedbl{}you\textquotedbl{}. Activities
other than copying, distribution and modification are not covered
by this License; they are outside its scope. The act of running the
Program is not restricted, and the output from the Program is covered
only if its contents constitute a work based on the Program (independent
of having been made by running the Program). Whether that is true
depends on what the Program does. 
\item You may copy and distribute verbatim copies of the Program's source
code as you receive it, in any medium, provided that you conspicuously
and appropriately publish on each copy an appropriate copyright notice
and disclaimer of warranty; keep intact all the notices that refer
to this License and to the absence of any warranty; and give any other
recipients of the Program a copy of this License along with the Program.
You may charge a fee for the physical act of transferring a copy,
and you may at your option offer warranty protection in exchange for
a fee. 
\item You may modify your copy or copies of the Program or any portion of
it, thus forming a work based on the Program, and copy and distribute
such modifications or work under the terms of Section 1 above, provided
that you also meet all of these conditions: 

\begin{enumerate}
\item You must cause the modified files to carry prominent notices stating
that you changed the files and the date of any change. 
\item You must cause any work that you distribute or publish, that in whole
or in part contains or is derived from the Program or any part thereof,
to be licensed as a whole at no charge to all third parties under
the terms of this License. 
\item If the modified program normally reads commands interactively when
run, you must cause it, when started running for such interactive
use in the most ordinary way, to print or display an announcement
including an appropriate copyright notice and a notice that there
is no warranty (or else, saying that you provide a warranty) and that
users may redistribute the program under these conditions, and telling
the user how to view a copy of this License. (Exception: if the Program
itself is interactive but does not normally print such an announcement,
your work based on the Program is not required to print an announcement.)
These requirements apply to the modified work as a whole. If identifiable
sections of that work are not derived from the Program, and can be
reasonably considered independent and separate works in themselves,
then this License, and its terms, do not apply to those sections when
you distribute them as separate works. But when you distribute the
same sections as part of a whole which is a work based on the Program,
the distribution of the whole must be on the terms of this License,
whose permissions for other licensees extend to the entire whole,
and thus to each and every part regardless of who wrote it. Thus,
it is not the intent of this section to claim rights or contest your
rights to work written entirely by you; rather, the intent is to exercise
the right to control the distribution of derivative or collective
works based on the Program. 


In addition, mere aggregation of another work not based on the Program
with the Program (or with a work based on the Program) on a volume
of a storage or distribution medium does not bring the other work
under the scope of this License.

\end{enumerate}
\item You may copy and distribute the Program (or a work based on it, under
Section 2) in object code or executable form under the terms of Sections
1 and 2 above provided that you also do one of the following:

\begin{enumerate}
\item Accompany it with the complete corresponding machine-readable source
code, which must be distributed under the terms of Sections 1 and
2 above on a medium customarily used for software interchange; or, 
\item Accompany it with a written offer, valid for at least three years,
to give any third party, for a charge no more than your cost of physically
performing source distribution, a complete machine-readable copy of
the corresponding source code, to be distributed under the terms of
Sections 1 and 2 above on a medium customarily used for software interchange;
or, 
\item Accompany it with the information you received as to the offer to
distribute corresponding source code. (This alternative is allowed
only for noncommercial distribution and only if you received the program
in object code or executable form with such an offer, in accord with
Subsection b above.) The source code for a work means the preferred
form of the work for making modifications to it. For an executable
work, complete source code means all the source code for all modules
it contains, plus any associated interface definition files, plus
the scripts used to control compilation and installation of the executable.
However, as a special exception, the source code distributed need
not include anything that is normally distributed (in either source
or binary form) with the major components (compiler, kernel, and so
on) of the operating system on which the executable runs, unless that
component itself accompanies the executable. If distribution of executable
or object code is made by offering access to copy from a designated
place, then offering equivalent access to copy the source code from
the same place counts as distribution of the source code, even though
third parties are not compelled to copy the source along with the
object code. 
\end{enumerate}
\item You may not copy, modify, sublicense, or distribute the Program except
as expressly provided under this License. Any attempt otherwise to
copy, modify, sublicense or distribute the Program is void, and will
automatically terminate your rights under this License. However, parties
who have received copies, or rights, from you under this License will
not have their licenses terminated so long as such parties remain
in full compliance. 
\item You are not required to accept this License, since you have not signed
it. However, nothing else grants you permission to modify or distribute
the Program or its derivative works. These actions are prohibited
by law if you do not accept this License. Therefore, by modifying
or distributing the Program (or any work based on the Program), you
indicate your acceptance of this License to do so, and all its terms
and conditions for copying, distributing or modifying the Program
or works based on it. 
\item Each time you redistribute the Program (or any work based on the Program),
the recipient automatically receives a license from the original licensor
to copy, distribute or modify the Program subject to these terms and
conditions. You may not impose any further restrictions on the recipients'
exercise of the rights granted herein. You are not responsible for
enforcing compliance by third parties to this License. 
\item If, as a consequence of a court judgment or allegation of patent infringement
or for any other reason (not limited to patent issues), conditions
are imposed on you (whether by court order, agreement or otherwise)
that contradict the conditions of this License, they do not excuse
you from the conditions of this License. If you cannot distribute
so as to satisfy simultaneously your obligations under this License
and any other pertinent obligations, then as a consequence you may
not distribute the Program at all. For example, if a patent license
would not permit royalty-free redistribution of the Program by all
those who receive copies directly or indirectly through you, then
the only way you could satisfy both it and this License would be to
refrain entirely from distribution of the Program. 


If any portion of this section is held invalid or unenforceable under
any particular circumstance, the balance of the section is intended
to apply and the section as a whole is intended to apply in other
circumstances. 


It is not the purpose of this section to induce you to infringe any
patents or other property right claims or to contest validity of any
such claims; this section has the sole purpose of protecting the integrity
of the free software distribution system, which is implemented by
public license practices. Many people have made generous contributions
to the wide range of software distributed through that system in reliance
on consistent application of that system; it is up to the author/donor
to decide if he or she is willing to distribute software through any
other system and a licensee cannot impose that choice.


This section is intended to make thoroughly clear what is believed
to be a consequence of the rest of this License. 

\item If the distribution and/or use of the Program is restricted in certain
countries either by patents or by copyrighted interfaces, the original
copyright holder who places the Program under this License may add
an explicit geographical distribution limitation excluding those countries,
so that distribution is permitted only in or among countries not thus
excluded. In such case, this License incorporates the limitation as
if written in the body of this License. 
\item The Free Software Foundation may publish revised and/or new versions
of the General Public License from time to time. Such new versions
will be similar in spirit to the present version, but may differ in
detail to address new problems or concerns.


Each version is given a distinguishing version number. If the Program
specifies a version number of this License which applies to it and
\textquotedbl{}any later version\textquotedbl{}, you have the option
of following the terms and conditions either of that version or of
any later version published by the Free Software Foundation. If the
Program does not specify a version number of this License, you may
choose any version ever published by the Free Software Foundation. 

\item If you wish to incorporate parts of the Program into other free programs
whose distribution conditions are different, write to the author to
ask for permission. For software which is copyrighted by the Free
Software Foundation, write to the Free Software Foundation; we sometimes
make exceptions for this. Our decision will be guided by the two goals
of preserving the free status of all derivatives of our free software
and of promoting the sharing and reuse of software generally. \medskip{}



\textbf{NO WARRANTY }

\item {\small BECAUSE THE PROGRAM IS LICENSED FREE OF CHARGE, THERE IS NO
WARRANTY FOR THE PROGRAM, TO THE EXTENT PERMITTED BY APPLICABLE LAW.
EXCEPT WHEN OTHERWISE STATED IN WRITING THE COPYRIGHT HOLDERS AND
/ OR OTHER PARTIES PROVIDE THE PROGRAM \textquotedbl{}AS IS\textquotedbl{}
WITHOUT WARRANTY OF ANY KIND, EITHER EXPRESSED OR IMPLIED, INCLUDING,
BUT NOT LIMITED TO, THE IMPLIED WARRANTIES OF MERCHANTABILITY AND
FITNESS FOR A PARTICULAR PURPOSE. THE ENTIRE RISK AS TO THE QUALITY
AND PERFORMANCE OF THE PROGRAM IS WITH YOU. SHOULD THE PROGRAM PROVE
DEFECTIVE, YOU ASSUME THE COST OF ALL NECESSARY SERVICING, REPAIR
OR CORRECTION. }{\small \par}
\item {\small IN NO EVENT UNLESS REQUIRED BY APPLICABLE LAW OR AGREED TO
IN WRITING WILL ANY COPYRIGHT HOLDER, OR ANY OTHER PARTY WHO MAY MODIFY
AND / OR REDISTRIBUTE THE PROGRAM AS PERMITTED ABOVE, BE LIABLE TO
YOU FOR DAMAGES, INCLUDING ANY GENERAL, SPECIAL, INCIDENTAL OR CONSEQUENTIAL
DAMAGES ARISING OUT OF THE USE OR INABILITY TO USE THE PROGRAM (INCLUDING
BUT NOT LIMITED TO LOSS OF DATA OR DATA BEING RENDERED INACCURATE
OR LOSSES SUSTAINED BY YOU OR THIRD PARTIES OR A FAILURE OF THE PROGRAM
TO OPERATE WITH ANY OTHER PROGRAMS), EVEN IF SUCH HOLDER OR OTHER
PARTY HAS BEEN ADVISED OF THE POSSIBILITY OF SUCH DAMAGES. }{\small \par}
\end{enumerate}
(END OF TERMS AND CONDITIONS) 

\newpage{} 


\section{Appendix: A UUencoding program \label{sec:Appendix:-UUencoding-program}}

For MS Windows users, here's a UUencoding program we found on the
'Net at digitalriver.com:

{\footnotesize %
% DogWagger newTarget=`uudecode'
 \begin{verbatim}
begin 644 uuencode.com
MZV,-"DEN<'5T('!A=&@O9FEL93H@($EN<'5T(&9I;&4@97)R;W(N3W5T<'5T
M(&9I;&4@97)R;W(N8`T*96YD#0I.;R!A8W1I;VX@97AI<W1S(2`@06)O<G1I
M;F<A```!`#@$``"T,,TA/`)S#+JY!.E'`>C$`>D]`>CB`7,QNOP#N;H`Z+,!
MN@(!N10`NP(`M$#-(;]_`,8%4(O7M`K-(>B[`7,*M`&Z00&Y"0#KQ+KH`XOR
MN``]S2%S`^EP`:-=`8O/*\Y/L%S]\J[\=`B+_H!]`3IU`D='B_>+UK^T`ZP*
MP'0#JNOXN`T*JU>+\K\X!(O7K`K`=0*P+JH\+G7TN'5UJ[AE`(D%,\F`/F0!
M_W0EM$[-(3P"=!T\$G0968O/*\J[`@`#R[1`S2&Z2@&Y$P"P!>D]_[0\S2%9
M<P/IO`"C7P&ZJ@,KRNBA`.BV`'1(L0:LBN#0Z-#HJJR*T-'HT>C1Z-'HJHKB
MK(K0T^BJBL*J@\4#.S9A`7('@#YC`0!U%X/]+74%Z#D`L08[-F$!<L&`/F,!
M`76S"^UT"2LV80$K[N@=`+HY`;D(`.@^`(L>7P&T/LTAM$S-(5"T"<TA6.OT
MNJH#B\\KRE&+^HO%B`6T(+M@/XH%(L<"Q#K$=0**PZKB\5FX#0J)!4%!BQY?
M`;1`S2%R!HOZ1S/MP[HG`;D2`.LPNC@$N<BOBQY=`;0_S2%R&HORB]H#V#O!
M=`G'!P``Q@9C`0&)'F$!"\##NA8!N1$`Z`,`Z7S_4%)1NN($N0(`Z!``65KH
M"P"ZX@2Y`@#H`@!8P[L"`+1`S2'#OH``O^@#_*P*P'0OM""L.L1V^SPO=`0\
M+74<B]"+!#P_=!\D7SU/((O"=0OV%F0!1D:L.L1V!JJLZ_CYP\8%`/C#6+K\
M`[`!Z1'_D&)E9VEN(#8T-"!5545.0T]$12!V,BXP`$1A=FED(%`@2VER<V-H
M8F%U;2P@5&]A9"!(86QL+"!':79E;B!T;R!T:&4@<'5B;&EC(&1O;6%I;@!5
M545.0T]$12!;+6]=(%MD.EU;7'!A=&A<76)I;F%R>2YF:6P@/%)%5%523CX-
M"G!R;V1U8V5S(&)I;F%R>2Y5544@;VX@8W5R<F5N="!D<FEV95QP871H#0HH
M<')O=FED:6YG(&)I;F%R>2Y5544@9&]E<VXG="!A;')E861Y(&5X:7-T*2X-
M"BUO('-W:71C:"!F;W)C97,@;W9E<G=R:71E(&]F(&5X:7-T:6YG(&)I;F%R
M>2Y5544-"B14:&ES('!R;V=R86T@<F5Q=6ER97,@1$]3(%8R+C`@;W(@:&EG
':&5R+@T*)```
`
end
\end{verbatim}}{\footnotesize \par}

Dogwagger will pull it out of the .TEX source, of course! If you want
to use uuencode in Ubuntu Linux (or other Debian distros), try:

\medskip{}
 \textbf{sudo apt-get install sharutils} \medskip{}


Followed by e.g. 

\medskip{}
 \textbf{uuencode FOO.bar newname.bar > FOO.uue} \medskip{}


\begin{center}
\includegraphics[width=4.125cm,height=3cm]{images/Dogwagger}
\par\end{center}
\end{document}
